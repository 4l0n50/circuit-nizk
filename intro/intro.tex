% !TEX root = ../main-circuit-nizk.tex

\newcommand{\setsize}{t}

In this work we construct a NIZK argument of knowledge (NIZK-AoK) for the language
\[
\mathsf{CircuitSat}:=\left\{
	C : \exists \vecb{x}\in\Z_p^m \text{ s.t. } C \text{ is an algebraic circuit and } C(\vecb{x})=1
	\right\},
\]
with proof size $\kappa+\Theta(\mathrm{depth}(C))$ elements of a bilinear group, where $\kappa$ is the size of a proof of knowledge of $\vecb{x}$. In the case of binary circuits, i.e.~$p=2$, we have that $\kappa=2|\vecb{x}|+O(1)$ using the techniques of \cite{AC:GonHevRaf15}. In general, $\kappa$ sould be independent from the circuit.
%
%We can equivalently see our result as what we call a \emph{conditonal} AoK (cAoK and NIZK-cAoK when it also has the NIZK properties). In a cAoK soundness is only guaranteed when one assumes that the adversary knows some secret $\vecb{x}$. This implies that now the 

%We do so by constructing a QA-NIZK proof system for the language
%\[
%\mathsf{CircuitSat}_{ck}:=\left\{\begin{array}{l}
%([\grkb{\zeta}_1]_1,\ldots,[\grkb{\zeta}_n]_1,C):\exists x_1,\ldots,x_n\in\Z_q,\rho_1,\ldots,\rho_n\in\Z_q \text{ s.t. } \\
%C(x)=1 \text{ and } \forall i\in [n]\ [\grkb{\zeta}_i]_1=\GS.\Com_{ck}([x_i]_1;\rho_i)
%\end{array}\right\},
%\]
%with proof size $\Theta(\mathrm{depth}(C))$.

We organize the circuit gates by level, where level $\ell$ is formed by the gates at distance $\ell$ from the output gate. For example, the $d$-th level, where $d:=\mathrm{depth}({C})$, contain the gates whose inputs are only elements from the circuit input $\vecb{x}$ and the $0$-th level contains the unique gate whose output is the output of the circuit.

To each level we might associate a vector of degree 2 polynomials $\vecb{p}_\ell \in \Z_q^{n_\ell}[W_1,\ldots,W_{m_\ell}]$ s, where $m_\ell\in\mathbb{N}$ is the number of inputs of level $\ell$ and $n_\ell\in\mathbb{N}$ is the number of outputs (or, equivalently the number of gates) of level $\ell$. For each level $\ell$ it must hold that $\vecb{w}_\ell=\vecb{p}_\ell(\vecb{w}_{\ell-1})$ is the vector of outputs of all gates at level $\ell$ when its input is $\vecb{w}_{\ell-1}$.
W.l.o.g.~we assume that the outputs of level $\ell-1$ are all the inputs of level $\ell$, since we might add identity gates. For example, suppose that $x_3$ is an input of level $d-1$, then we add an identity gate at level $d$ and thus $\vecb{p}_d=(\ldots,x_3,\ldots)$.
 Therefore, it must hold that $ m_\ell = n_{\ell-1}$ and that for every $\vecb{x}\in\Z_p^m$
$$
C(\vecb{x}) = \vecb{p}_{d}\circ\vecb{p}_{d-1}\circ\ldots\circ \vecb{p}_0 (\vecb{x})
$$

We work on asymmetric bilinear groups and our construction is built from the following primitives:
\begin{enumerate}
\item Homomorphic commitment schemes for vectors of integers in $\Z_q^m$ with randomness in $\Z_q^r$ and the following properties:
\begin{enumerate}
	\item Commitment keys are matrices over one of the base groups $$ck=[\matr{G}_0\cat\matr{G}_1]_s\in\GG_s^{k\times(m+r)}$$ and $k>r$.
	\item Commitments are defined by $\Com_{ck}(\vecb{x};\vecb{r})=[\matr{G}_0]_s\vecb{x}+[\matr{G}_1]_s\vecb{r}$.		\item Whenever $k=m+r$, $\Com$ defines perfectly binding commitments (or equivalently $\matr{G}$ is invertible) and 	                   there is a QA-NIZK AoK of an oppening. In this case we will say that is a knowledge commitment and write $\mathsf{KCom}$ instead of $\Com$.
\item One might sample $ck$ together wit a trapdoor $t$ which allows to decide whether some commitment opens to $0$ or not.
	\item there is a computationally indisintguishable way of sampling the commitment keys where $\mathrm{Im}(\matr{G}_0)\subseteq\mathrm{Im}\mathrm(\matr{G}_1)$.
\end{enumerate}
\item A constant size QA-NIZK argument for the following language
$$
\mathcal{L}_{\mathsf{deg}\mbox{-}2,ck,ck'}(\vecb{p}):=\left\{[\vecb{c}]_s,[\vecb{c}']_s:
\begin{array}{l}
		\text{knowledge of } \vecb{x} \text{ s.t. }
		{[\vecb{c}]_1=\Com_{ck}(\vecb{x})}
		\Longrightarrow\\
		{[\vecb{c}']_1=\mathsf{Com}_{ck'}(\vecb{p}(\vecb{x}))}
	\end{array}\right\},
$$
for some $\vecb{p}\in\Z_p^n[X_1,\ldots,X_m]$ of degree at most 2. The commitments keys are of size $ck\in\GG_s^{k\times(m+r)},ck'\in\GG_s^{k\times(n+r)}$ and $k$ is a constant polynomial in the security parameter (independent of both $m$ and $n$).


{\color{red} I don't know if it would be a good idea to introduce a notion of conditional argument (or proof) of knowledge, where the soundness reduction has access to an oppening of the first commitment.}
\item QA-NIZK arguments of membership in linear spaces as defined in \cite{EC:KilWee15}.
\end{enumerate}

Lets quickly describe how our proof system is built from this primitives. The CRS is composed by the commitent key $ck_\mathsf{PoK}$ for commitments of type (b) and the CRS $\crs_\mathsf{PoK}$ for proving knowledge of an oppening of these commitments. It also contains commtiment keys $ck_{d+1},ck_d\ldots,ck_0$ for commitments and CRS's $\crs_d,\ldots,\crs_0$ for proving membership in $\mathcal{L}_{\mathsf{deg}\mbox{-}2,ck_{d-1},ck_d}(\vecb{p}_d),\allowbreak\ldots,\allowbreak\mathcal{L}_{\mathsf{deg}\mbox{-}2,ck_1,ck_0}(\vecb{p}_0)$, respectively. These commitment keys define commitments of size $k=r+1=O(1)$, i.e.~independent of the size of $\vecb{x}$, and for $\crs_0$, $k=m+r=r+1$ (there is a unique output) and define perfectly binding commtiments. Finally, the CRS contains $\crs_\mathsf{lin}$ for proving membersip in a linear space.

Given some $\vecb{x}$ such that $C(\vecb{x})= 1$, the prover computes the proof as follows. It computes a knowledge commitment $[\vecb{c}]_1\gets\mathsf{KCom}(\vecb{x}$), together with $\pi$ an AoK of $\vecb{x}$. It computes constant-size commtiments $[\vecb{c}_\ell]_1\gets\Com_{ck_\ell}(\vecb{p}_d\circ\ldots\circ\vecb{p}_\ell(\vecb{x}))$  with proofs $\pi_\ell$ that $([\vecb{c}_{\ell-1}]_1,[\vecb{c}_\ell]_1)\in\mathcal{L}_{\mathsf{deg}\mbox{-}2,ck_{\ell-1},ck_\ell}(\vecb{p}_\ell)$, for $1\leq \ell\leq d$,. It additionally computes $[\vecb{c}_{d+1}]_1\gets\Com_{ck_{d+1}}(\vecb{x})$ and provides a proof $\theta_\mathsf{lin}$ that $[\vecb{c}]_1$ and $[\vecb{c}_{d+1}]_1$ can be oppened to the same value and $\pi_\mathsf{lin}$ that $[\vecb{c}_0]_1$  can be oppened to 1 (both $\theta_\mathsf{lin}$ and $\pi_\mathsf{lin}$ can be easily constructed from a QA-NIZK argument of membership in a linear space as shown in \cite{AC:GonHevRaf15}).

An intuitive reason of why this proof system is sound is as follows. Suppose an adversary produces a proof $\pi$ for a circuit $C$ such that is impossible to extract from $\pi$ some $\vecb{x}$ s.t.~$C(\vecb{x})=1$. In particular, let $\vecb{x}$ the oppening of $c$ which can be extracted from $\pi_\mathsf{PoK}$, then $C(\vecb{x})=0$. Soundness of $\theta_\mathsf{lin}$ implies that $[\vecb{c}]_1$ and $[\vecb{c}_{d+1}]_1$ share a common oppening. Since $[\vecb{c}]_1$ is perfectly binding, the unique oppening they can share is $\vecb{x}$ and thus $[\vecb{c}_{d+1}]=\Com_{ck_{d+1}}(\vecb{x})$. Similarly, $[\vecb{c}_0]_1$ has a unique oppening since $m=1$ and thus $k=m+r$. Then by the soundness of $\pi_\mathsf{lin}$, its unique opening must be equal to 1.
Let $\vecb{w}_\ell := \vecb{p}_d\circ\ldots\circ \vecb{p}_\ell(\vecb{x})$, $0\leq\ell\leq d$, and let $\ell^*$ the larger index between $0$ and $d$ such that $[\vecb{c}_{\ell^*}]\neq\Com_{ck_{\ell^*}}(\vecb{w}_{\ell^*};\vecb{\rho})$ for every $\vecb{\rho}$. Note that such $\ell^*$ exists since at least $[\vecb{c}_0]\neq\Com_{ck_0}(\vecb{w}_0 = C(\vecb{x})=0)$ by soundness of $\pi_\mathsf{lin}$. Hence, from $[\vecb{c}]_1$ we can extract $\vecb{x}$ from which we can compute $\vecb{w}_{\ell^*-1} = \vecb{p}_d\circ\ldots\circ \vecb{p}_{\ell^*-1}(\vecb{x})$, which an oppening of $[\vecb{c}_{\ell^*-1}]_1=\Com_{ck_{\ell^*-1}}(\vecb{w}_{\ell^*-1})$, and thus $([\vecb{c}_{\ell^*-1}],[\vecb{c}_{\ell^*}])\notin\mathcal{L}_{\mathsf{deg}\mbox{-}2,ck_{\ell^*-1},ck_{\ell^*}}(\vecb{p}_{\ell^*})$ which violates the soundness of $\pi_{\ell^*}$.

\section{The primitives}

\subsection{Homomorphic Commitments.}
Both Groth-Sahai and Pedersen commitments are special cases of the following general commitment scheme
\begin{align*}
&ck:=[\matr{G}]_s=[\matr{G}_0|\matr{G}_1]\in\GG_s^{k\times (m+r)}, \quad \matr{G}_0 :=\vecb{u}_0\matr{A}
& \mathsf{Com}_{ck}(\vecb{x};\vecb{\rho})=[\matr{G}_0]_s\vecb{x}+[\matr{G}_1]_s\vecb{\rho}.
\end{align*}
Groth-Sahai commitments based on the $\dist_{k,r}$-MDDH assumption correspond to the case $m=1$, $\matr{G}_1\gets\dist_{k,r}$ and $\matr{G}_0\in\Z_q^k$ such that $\matr{G}_0\notin\mathrm{Im}(\matr{G}_1)$. Pedersen commitments correspond to the case $k=r=1$ and $\matr{G}\gets\Z_q^{1\times(m+1)}$, which defines perfectly hiding commitments.

{\color{red} Generalize to any Matrix Assumption?}
We will consider the case $\matr{G}\gets\Z_q^{k\times r}$ which has been called \emph{somewhere statiscally binding}. 
This commitments clearly satisfy properties (a) and (b). We now proceed to prove the also satisfy (c), (d), and (e).
\begin{enumerate}[label=(\alph*)]
\setcounter{enumi}{2}
\item If $k=n+r$, with ovewhelming probability $\matr{G}$ is invertible and thus $[\vecb{c}]_s=\Com_{ck}(\vecb{x};\vecb{\rho})$ defines a unique oppening $\vecb{x}$. When $p=2$, i.e.~$\vecb{x}\in\bits^m$, it suffices to prove that indeed $\vecb{x}\in\bits^m$ which can be done using the constant size proof from \cite{AC:GonHevRaf15}. In the general case, we can use range proofs from \cite{ACNS:GonRaf16} which are in fact arguments of knowledge. With those techniques we get an AoK of size $m\cdot\mathrm{size}(\text{range proof in }[0,2^p-1])=mO(p/\log p)$.

We might even construct a constant size AoK for the case $k=O(1)$ but at the cost of relying its security on non falsifiable assumptions. Indeed, pick any SNARK for NP (e.g.~\cite{EC:Groth17}) and give an AoK of an oppening. {\color{red} Can we get a more direct (without reduction to a circuit nor a quadratic arithmethic program) SNARK by means of a knowledge assumption more related to our setting.?}
\item Define the trapdoor $\vecb{t}\in\Z_q^k$ as a random vector in $\mathrm{Ker}(\matr{G}_1^\top)$. 
%Let $\matr{U}\in\Z_q^{k\times(k-r)}$ an orthonormal basis of $\mathrm{Ker}(\matr{G}_1^\top)$ and hence $\vecb{t}=\matr{U}\vecb{\alpha}$ for some random $\vecb{\alpha}\in\Z_q^{k-r}$.
Clearly, if $[\vecb{c}]_s=\Com_{ck}(\vecb{0};\vecb{\rho})$ for some $\vecb{\rho}$, then $\vecb{t}^\top\vecb{c}=0$. If $[\vecb{c}]\neq\Com_{ck}(\vecb{0};\vecb{\rho})$ for any $\rho$, then $\vecb{c}=\sum_{i\leq m}\vecb{g}_i\vecb{x}+\sum_{i>m}\vecb{g}_i\rho_i$, where $\vecb{g}_i$ is the $i$-th column of $\matr{G}$, and there is some $i^*\leq m$ s.t. $x_{i^*}\neq 0$. Since $\vecb{t}$ is independent from $\vecb{g}_{i^*}$, then $\vecb{t}^\top\vecb{g}_{i^*}\neq 0$ with overwhelming probability and then $\vecb{t}^\top\vecb{c}\neq 0$. 
\item We simply pick $\matr{G}_0:=\matr{G}_1\matr{W}$ for some $\matr{W}\gets\Z_q^{r\times m}$. Indistinguishablity follows from the DDH assumption in $\GG_s$ and clearly $\mathrm{Im}(\matr{G}_0)\subseteq\mathrm{Im}(\matr{G}_1)$.
%$\notin\mathrm{Im}(\matr{G}_1)$ and $\matr{G}_0$ is full rank. Let $\matr{T}\in\Z_q^{(k-r)\times k}$ such that $\matr{T}\matr{G}_1=\vecb{0}$ and $\matr{T}\matr{G}_0=\matr{I}_{m}\dsum\matr{0}_{r}$. Indeed, define $\vecb{k}_i$ as some verctor in the kernel of $\matr{G}_{-i}^\top$, where $\matr{G}_{-i}\in\Z_q^{k\times(m+r-1)}$ is exactly as $\matr{G}$ but with its $i$-th column removed. Denote by $\vecb{g}_i$ the $i$-th column of $\matr{G}$, since $\vecb{t}_i$ is independent from $\vecb{g}_i$, then $\langle\vecb{t},\vecb{g}_i\rangle \neq 0$ with overwhelming probability. Define now $\vecb{t}=\vecb{k}_i/\langle\vecb{k}_i,\vecb{g}_i\rangle$ and define $\matr{T}:=\vecb{t}_1^\top\dsum\ldots\vecb{t}_m^\top$. For any $[\vecb{c}]_s$ there is a unique oppening $\vecb{x}$ such that  $\matr{T}\matr{G}_1=\matr{0}$ and $\matr{T}\matr{G}_1=\matr{I}_m\matr{0}_r$ as desired.
\end{enumerate}

\subsection{QA-NIZK AoK for Quadratic Polynomials}
We construct a constant size QA-NIZK argument that, given two commtiments  (of the type defined above) and knowledge of $\vecb{x}$, and opening of the former commitment, the the prover must know an oppening of the latter and it must be equal.
In turn, this QA-NIZK argument is constructed from the follwing primitives:
\begin{enumerate}
\item A QA-NIZK argument for the following language
$$
\mathcal{L}_{\mathsf{prod},ck_1,ck_2}=\left\{[\vecb{a}]_1,[\vecb{b}]_2,[\vecb{c}]_1:
	\begin{array}{c}
		[\vecb{a}]_1=\mathsf{Com}_{ck_1}(\vecb{x})\text{ and }
		{[\vecb{b}]_2=\mathsf{Com}_{ck_2}(\vecb{y})}\\
		\Longrightarrow
		[\vecb{c}]_1=\Com_{ck_3}(\vecb{x}\otimes\vecb{y})
	\end{array}\right\},
$$
where $\vecb{x}\in\Z_q^m,\vecb{y}\in\Z_q^n,\vecb{x}\otimes\vecb{y}\in\Z_q^{mn}$, $e(ck_3,[\matr{I}]_{2})=ck_1\otimes ck_2\allowbreak\in\GG_T^{k_1k_2\times(m+r_1)(n+r_2)}$, and $\otimes$ denote the kroenecker product between matrices with entries in $\Z_q$ or $\GG_s$, where multiplication is replaced by the pairing function when necessary.
\item A QA-NIZK argument for the language
$$
\Lang_{\mathsf{refresh},ck,ck'} = \left\{[\vecb{c}]_1,[\vecb{c}']_1:
	\begin{array}{l} \text{knowledge of } \vecb{x} \text{ s.t. }
		{[\vecb{c}]_1=\Com_{ck_1\otimes ck_2}(\vecb{x})}
		\Longrightarrow\\
		{[\vecb{c}']_1=\mathsf{Com}_{ck'}(\vecb{x})}
	\end{array}\right\},
$$
\item A QA-NIZK argument for the language
$$
\mathcal{L}_{\eq,ck,ck_1,ck_2,\matr{\Gamma}_1,\matr{\Gamma}_2} = \left\{[\vecb{c}]_1,[\vecb{a}']_1,[\vecb{b}']_2:
	\begin{array}{l}
		\text{knowledge of } \vecb{x} \text{ s.t. }[\vecb{c}]_1=\Com_{ck}(\vecb{x})\\
		\Longrightarrow
		[\vecb{a}']_1=\mathsf{Com}_{ck_1}(\matr{\Gamma}_1\vecb{x})\text{ and }\\
		{[\vecb{b}']_2=\mathsf{Com}_{ck_2}(\matr{\Gamma}_2\vecb{x})}
	\end{array}\right\},
$$
\end{enumerate}

Note that any polynomial $\vecb{p}\in\Z_q^n[X_1,\ldots,X_m]$ can be written as $\matr{\Gamma}_1(\vecb{X}\otimes\vecb{X})+\matr{\Gamma}_2\vecb{X}$, since is $\vecb{p}$ a linear combination of degree two monomials (i.e.~$\matr{\Gamma}_1(\vecb{X}\otimes\vecb{X})$) and degree one monomials (i.e.~$\matr{\Gamma}_2\vecb{X}$).
We prove that $[\vecb{c}']_1=\Com_{ck'}(\vecb{p}(\vecb{x}))=\Com_{ck'}(\matr{\Gamma}_1(\vecb{X}\otimes\vecb{X})+\matr{\Gamma}_2\vecb{X})$ from knowledge of $\vecb{x}$ s.t. $[\vecb{c}]_1 = \Com_{ck}(\vecb{x})$ as follows:
\begin{enumerate}
\item Prove that $[\vecb{a}]_1:=[\vecb{c}]_1$,  $[\vecb{0}]_1$ and $[\vecb{b}]_2=\Com_{ck_2}(\vecb{x})$ belongs to $\Lang_{\eq,ck,[\matr{0}]_1,ck_2,\matr{0},\matr{I}_{m}}$.
\item Prove that $[\vecb{a}]_1,[\vecb{b}]_2$ and $[\vecb{d}]:=\Com_{ck_3}(\vecb{x}\otimes\vecb{x})$ belongs to $\Lang_{\mathsf{prod},ck_1,ck_2}$, where $ck_1:=ck$ and $e(ck_3,[\matr{I}]_2)=ck_1\otimes ck_2$.
\item A proof that $[\vecb{d}]_1$ and $[\vecb{d}']_1:=\Com_{ck'}(\vecb{x}\otimes\vecb{x})$ belongs to $\Lang_{\mathsf{refresh},ck_3,ck'}$ for $ck'$ a random commitment key for vectors of size $n$ .
\item A proof that $[\vecb{d}]_1\dsum[\vecb{c}]_1=\Com_{ck'\oplus ck}((\vecb{x}\otimes\vecb{x})\dsum\vecb{x})$ and $[\vecb{c}']_1=\Com_{ck'}(\matr{\Gamma_1}(\vecb{x}\otimes\vecb{x})+\matr{\Gamma}_2\vecb{x}),[\vecb{0}]_2$ belongs to $\Lang_{\eq,ck'\oplus ck,ck',[\matr{0}]_2,\matr{\Gamma}_1\oplus\matr{\Gamma}_2,\matr{0}}$, where
$$
ck'\oplus ck := \left[\begin{matrix}
\matr{G}'_0 & \matr{0}      & \matr{G}'_1 &  \matr{0} \\
\matr{0}      & \matr{G}_0  & \matr{0}      & \matr{G}_1
\end{matrix}\right]_1,
\text{ where } ck'=[\matr{G}'_0\cat\matr{G}'_1]_1, ck=[\matr{G}_0\cat\matr{G}_1]_1
$$
and
$
\matr{\Gamma}_1\oplus\matr{\Gamma}_2 := \begin{pmatrix}
\matr{\Gamma}_1 & \matr{0}      \\
\matr{0}      & \matr{\Gamma}_2  
\end{pmatrix}.
$
\end{enumerate}

\subsubsection{QA-NIZK for Tensor Product Relations (1).}
Using the commitments defined before it is easy to derive commitments to $\vecb{x}\otimes\vecb{y}$ from commitments to $\vecb{x}\in\Z_q^m$ and $\vecb{y}\in\Z_q^n$, as follows
$$
\Com_{ck_3}(\vecb{\vecb{x}}\otimes\vecb{y};\vecb{\rho}_3):=\Com_{ck_1}(\vecb{x};\vecb{\rho}_1)\otimes\Com_{ck_2}(\vecb{y};\vecb{\rho}_2),
$$
where $ck_2:=[\matr{H}_0|\matr{H}_2]_1,ck_3=[\matr{G}\otimes\matr{H}]_T$ and
$$\vecb{\rho}_3=\pmatri{\vecb{0}_m\\\vecb{\rho}_1}\otimes\pmatri{\vecb{y}\\\frac{1}{2}\vecb{\rho}_2}+\pmatri{\vecb{x}\\\frac{1}{2}\vecb{\rho}_1}\otimes\pmatri{\vecb{0}_n\\\vecb{\rho}_2}$$ ({\color{red} check if $\vecb{\rho_3}$ is correct}).

This approach has the disadvantage that once we compute $[\vecb{c}]_T=\allowbreak\Com_{ck_3}(\vecb{x}\otimes\vecb{y})$ we are stucked in the target group --- hence no more multiplications nor interesting NIZK proofs can be made --- and we need to bring back $ck_3$ from $\GG_T$ to $\GG_1$. We can \emph{bootstrap} commitment $[\vecb{c}]_T$  by bringing it the base group $\GG_1$ (or similarly to $\GG_2$) and requiring the verifier to check that
\begin{equation}
e([\vecb{a}]_1,[\vecb{b}]_2)=e([\vecb{c}]_1,[\matr{I}]_{2}).\label{eqn:bootstrap}
\end{equation}
{\color{red} Maybe a GS proof? }

The QA-NIZK argument in this case is very simple. The CRS must include $ck_3$ and the proof is simply $[\vecb{c}]_1=\Com_{ck_3}(\vecb{x}\otimes\vecb{y};\vecb{\rho_3})$ while the verifier checks that $(\ref{eqn:bootstrap})$ holds.
%
%Going a step forward, we will have to give two shares of $[\vecb{c}]_s$,  $[\vecb{c}']_1$ and $[\vecb{d}']_2$, such that $\vecb{c}=\vecb{c}'+\vecb{d}'$. We omit the ``primes'' in the shares and now the verifier checks that
%$$
%e([\vecb{a}]_1,[\vecb{b}]_2)=e([\vecb{c}]_1,[\matr{I}]_{2}) + e([\matr{I}]_{1},[\vecb{d}]_2).
%$$
%
%The first share is computed using commitment key $ck_{3,1} := [\matr{G}\otimes\matr{H}-\matr{Z}]_1$ and the second share is computed using commitment key $ck_{3,1} := [\matr{Z}]_2$, for $\matr{Z}\gets\Z_q^{k_1k_2\times mn}$.

\subsubsection{QA-NIZK AoK for commitment key refreshing (2).}
Given $[\vecb{c}]_1=\allowbreak\Com_{ck}(\vecb{x};\allowbreak \vecb{\rho})$, where $ck = ck_1\otimes ck_2$, and an AoK of $\vecb{x}$, we want to show that $[\vecb{c}']_1$ can be also oppened to $\vecb{x}$ but $ck'$ is a random commitment key.

To do so we will give a QA-NIZK argument that $\vecb{c}\dsum\vecb{c}'$ is in the linear span of
$$
\matr{J}:=
\begin{pmatrix}
\matr{G}_0\otimes\matr{H}_0 & \matr{G}_0\otimes\matr{H}_1 & \matr{G}_1\otimes\matr{H}_0& \matr{G}_1\otimes\matr{H}_1 & \matr{0}\\
\matr{G}'_0 & \matr{0} & \matr{0} & \matr{0} & \matr{G}'_0 
\end{pmatrix}
$$
However, the QA-NIZK argument only shows the existence of some $\vecb{w}$ such that $\vecb{c}\dsum\vecb{c}' = \matr{J}\vecb{w}$ but it might be the case that $\vecb{c}'$ still can't be oppened to $\vecb{x}$ --- i.e.~$\vecb{w}$ can't be $\vecb{x}$ appended with some other vector. We will show that this is not the case when $\vecb{x}$ is known in the security proof.

Assume that $[\vecb{c}]_1 = \Com_{ck}(\vecb{x};\vecb{\rho})$ but $[\vecb{c}']_1\neq \Com_{ck'}(\vecb{x};\vecb{\rho}')$ for any $\vecb{\rho}'$, and assume also that the adversary provides a valid proof $[\pi]_1$ for $[\vecb{c}\dsum\vecb{c}']_1$. Given knowledge of $\vecb{x}$, we can compute $[\vecb{c}^\dag]_1:=\Com_{ck}(\vecb{x};\vecb{0})$ and $[\vecb{c}^\ddag]:=\Com_{ck'}(\vecb{x};\vecb{0})$, and note that $\vecb{c}^\dag\dsum\vecb{c}^\ddag$ is in the immage of $\matr{J}$ and thus we can compute a proof $[\pi^\dag]_1$ for $[\vecb{c}^\dag\dsum\vecb{c}^\ddag]_1$. By the properties of the QA-NIZK arguments for linear spaces, we get that $[\pi-\pi^\dag]_1$ is a proof for $[\vecb{d}^\dag\dsum\vecb{d}^\ddag]_1$, where
$$[\vecb{d}^\dag]_1=[\vecb{c}-\vecb{c}^\dag]_1= \Com_{ck}(\vecb{0};\vecb{\rho})$$ and 
$$[\vecb{d}^\ddag]_1 = [\vecb{c}'-\vecb{c}^\ddag]\neq\Com_{ck}(\vecb{0},\vecb{\rho}^\ddag)$$ for any $\vecb{\rho}^\ddag$.

We will show that $\vecb{d}^\dag\dsum \vecb{d}^\ddag$ is not in the immage of $\matr{J}'$, such that $[\matr{J}']_1$ is computationally indistinguishable from $[\matr{J}]_1$.

Let $\vecb{u}_0,\vecb{u}_1,\vecb{v}_0,\vecb{v}_1,\vecb{u}'_0,\vecb{u}'_1$ randomly chosen from $\Z_q^k$. We compute $\matr{J}$ as before but now $ck_1,ck_2$ and $ck'$ are computed as follows
\begin{align}
&ck_1 = [\matr{G}_0|\matr{G}_1]_1 = [\vecb{u}_0\matr{A}_{0}|\vecb{u}_1\matr{A}_1]_1 \nonumber \\
&ck_2 = [\matr{H}_0|\matr{H}_1]_2 = [\vecb{v}_0\matr{B}_{0}|\vecb{v}_1\matr{B}_1]_2 \nonumber \\
&ck' =  [\matr{G}'_0|\matr{G}'_1]_1 = [\vecb{u}'_0(\matr{A}_0\otimes\matr{B}_0) + \vecb{u}_1\matr{C}_0|\vecb{u}_1\matr{C}_1]_1 \label{eq:ck-dist}
\end{align}
since $[\vecb{u}]_s\mu$, $\mu\gets \Z_q$, is indistinguishable from a random element in $\GG_s^k$ --- as long as the DDH assumption is hard in $\GG_s$ --- it follows that the new commitment keys are indistinguishable from the original ones.

There is still a technical problem with this approach: when using the DDH assumption in $\GG_2$ to change the distribution of $ck_2$ we can only compute $[\matr{J}]_2$ while we need to compute $[\matr{J}]_1$ to carry out the soundness proof. This problem has already arised and solved in \cite{AC:GonHevRaf15} and we use a similar solution in our final proof system. For the sake of clarity, for this intuitive explanation we just assume that $ck_1,ck_2$ and $ck'$ are sampled from (\ref{eq:ck-dist}) in the real game (although this will render impossible to prove zero-knowledge).\footnote{With symmetric bilinear groups this problen doesn't even exists, and in the soundness proof we might change $[\matr{J}]_1$ distribution without any problem.}

Going back to the problem of whether $\vecb{d}^\dag\dsum \vecb{d}^\ddag$ is in the immage of $\matr{J}$, we get that now this is not the case. Indeed, define $\vecb{u}_{i,j}:=\vecb{u}_i\otimes\vecb{v}_j$, $i,j\in\bits$, and note that matrix $\matr{J}$ is equal to
$$
\begin{pmatrix}
\vecb{u}_{0,0}(\matr{A}_0\otimes\matr{B}_0) & \vecb{u}_{0,1}(\matr{A}_0\otimes\matr{B}_1) & \vecb{u}_{1,0}(\matr{A}_1\otimes\matr{B}_0) & \vecb{u}_{1,1}(\matr{A}_1\otimes\matr{B}_1) & \vecb{0}\\
\vecb{u}'_0(\matr{A}_0\otimes\matr{B}_0) +\vecb{u}'_1\matr{C}_0 & \vecb{0} & \vecb{0} & \vecb{0} & \vecb{u}'_1\matr{C}_1
\end{pmatrix}
$$
and that $\vecb{d}^\dag\dsum\vecb{d}^\ddag$ can be written as
$$
\begin{pmatrix} \vecb{d}^\dag\\ \vecb{d}^\ddag \end{pmatrix}
=
\begin{pmatrix}
\vecb{u}_{0,1}\mu_{0,1} + \vecb{u}_{1,0}\mu_{1,0} + \vecb{u}_{1,1}\mu_{1,1}\\
\vecb{u}'_{0}\nu_{0} + \vecb{u}'_1\nu_1
\end{pmatrix},
\text{ where } \nu_0 \neq 0.
$$
Lets see that $\vecb{d}^\dag\dsum\vecb{d}^\ddag$ is not in the immage of $\matr{J}$ by showing that there aren't solutions to $\vecb{d}^\dag\dsum\vecb{d}^\ddag=\matr{J}(\vecb{w}_{0,0}\dsum\vecb{w}_{0,1}\dsum\vecb{w}_{1,0}\dsum\vecb{w}_{1,1}\dsum\vecb{w}_2)$. Indeed, suppose that
\begin{align}
\begin{pmatrix}
\vecb{u}_{0,1}\mu_{0,1} + \vecb{u}_{1,0}\mu_{1,0} + \vecb{u}_{1,1}\mu_{1,1}\\
\vecb{u}'_{0}\nu_{0} + \vecb{u}'_1\nu_1
\end{pmatrix}
=
\begin{pmatrix}
\sum_{i,j\in\bits}\vecb{u}_{i,j}(\matr{A}_i\otimes\matr{B}_j)\vecb{w}_{i,j}\\
\vecb{u}_0(\matr{A}_0\otimes\matr{B}_0)\vecb{w}_{0,0} + \vecb{u}'_1\matr{C}_0\vecb{w}_{0,0}+\vecb{u}'_1\matr{C}_1\vecb{w}_2.
\end{pmatrix}
\label{eq:d-li}
\end{align}
Given that $\vecb{u}_{0,0}$ is linearly independetn from $\{\vecb{u}_{0,1},\vecb{u}_{1,0},\vecb{u}_{1,1}\}$ and that $\vecb{u}_{0,0}$ doesn't appear on the left side of the first row of equation (\ref{eq:d-li}), it must hold that $(\matr{A}\otimes\matr{B})\vecb{w}_{0,0}=\vecb{0}$. Then, the second row is reduced to
$$
\vecb{u}'_{0}\nu_{0} + \vecb{u}'_1w_0\nu_1 = \vecb{u}'_1(\matr{C}_0\vecb{w}_{0,0}+\matr{C}_1\vecb{w}_2).
$$
Since $\vecb{u}'_0$ is linearly independent from $\vecb{u}'_1$, it must hold that $\nu_0=0$ but this contradicts the fact that $\vecb{c}'\neq\Com_{ck'}(\vecb{x};\vecb{\rho}')$ for all $\vecb{\rho}'$. We conclude that $\vecb{d}^\dag\dsum\vecb{d}^\ddag$ is not in the immage of $\matr{J}$ and $[\pi-\pi^\dag]$ is a proof of a false statement, contradicting the soundness of the QA-NIZK proof system for linear languages.

\subsubsection{QA-NIZK AoK of linear transformations (3)}
Given $[\vecb{c}]_1=\Com_{ck}(\vecb{x};\vecb{\rho})$ an an AoK of $\vecb{x}$ we want to show that $[\vecb{a}']_1$ can be oppened to $\matr{\Gamma}_1\vecb{x}$ and $[\vecb{b}']_2$ can be oppened to $\matr{\Gamma}_2$.

To do so we will use essentially the same techiniques used for (2). We give a QA-NIZK argument that $\vecb{c}\dsum\vecb{a}'\dsum\vecb{b}'$ is in the linear span of
$$
\matr{J}:=
\begin{pmatrix}
\matr{G}_0                              & \matr{G}_1 & \matr{0}      & \matr{0} \\
\matr{G}'_0\matr{\Gamma_1}& \matr{0}     & \matr{G}'_1 & \matr{0} \\
\matr{H}'_0\matr{\Gamma_2}& \matr{0}    & \matr{0}_1   & \matr{H}'_1 
\end{pmatrix},
$$
where $ck=[\matr{G}_0\cat\matr{G}_1]$, $ck=[\matr{G}'_0\cat\matr{G}'_1]$, and $ck=[\matr{H}'_0\cat\matr{H}'_1]$.
Again, the QA-NIZK argument only shows the existence of some $\vecb{w}$ such that $\vecb{c}\dsum\vecb{a}'\dsum\vecb{b}' = \matr{J}\vecb{w}$ but it might be the case that $\vecb{a}'$ or $\vecb{b}'$ still can't be oppened to $\matr{\Gamma}_1\vecb{x}$ or  $\matr{\Gamma}_2\vecb{x}$, respectively.

Assume that $[\vecb{c}]_1 = \Com_{ck}(\vecb{x};\vecb{\rho})$ but $[\vecb{a}']_1\neq \Com_{ck_1}(\matr{\Gamma}_1\vecb{x};\vecb{\rho}'_1)$ or $[\vecb{b}']_1\neq \Com_{ck_2}(\matr{\Gamma}_2\vecb{x};\vecb{\rho}'_1)$ for any $\vecb{\rho}'_1,\vecb{\rho}'_2$, and assume also that the adversary provides a valid proof $[\pi]_1$ for $[\vecb{c}\dsum\vecb{a}'\dsum\vecb{b}']_1$. Given knowledge of $\vecb{x}$, we can compute $[\vecb{c}^\dag]_1:=\Com_{ck}(\vecb{x};\vecb{0})$, $[\vecb{a}^\dag]:=\Com_{ck_1}(\matr{\Gamma}_1\vecb{x};\vecb{0})$, and $[\vecb{b}^\dag]_2:=\Com_{ck_2}(\matr{\Gamma}_1\vecb{x};\vecb{0})$, and note that $\vecb{c}^\dag\dsum\vecb{a}^\dag\dsum\vecb{b}^\dag$ is in the immage of $\matr{J}$ and thus we can compute a proof $[\pi^\dag]_1$ for $[\vecb{c}^\dag\dsum\vecb{a}^\dag\dsum\vecb{b}^\dag]_1$. By the properties of the QA-NIZK arguments for linear spaces, we get that $[\pi-\pi^\dag]_1$ is a proof for $[\vecb{d}^\dag\dsum\vecb{e}^\dag\dsum\vecb{f}^\dag]_1:=[\vecb{c}\dsum\vecb{a}'\dsum\vecb{b}']_1-[\vecb{c}^\dag\dsum\vecb{a}^\dag\dsum\vecb{b}^\dag]$, where
$$[\vecb{d}^\dag]_1 = \Com_{ck}(\vecb{0};\vecb{\rho}^\dag)\text{ for some }\vecb{\rho}^\dag$$
and
$$[\vecb{e}^\dag]_1 \neq\Com_{ck_1}(\vecb{0},\vecb{\rho}^\dag_1) \text{ for any }\vecb{\rho}^\dag_1$$ 
or
$$[\vecb{f}^\dag]_1 \neq\Com_{ck_2}(\vecb{0},\vecb{\rho}^\dag_2) \text{ for any }\vecb{\rho}_2^\dag.$$

We will show that $\vecb{d}^\dag\dsum \vecb{e}^\dag\dsum \vecb{f}^\dag$ is not in the immage of $\matr{J}'$, such that $[\matr{J}']_1$ is computationally indistinguishable from $[\matr{J}]_1$.

Let $\vecb{u}_0,\vecb{u}_1,\vecb{v}_0,\vecb{v}_1,\vecb{u}'_0,\vecb{u}'_1$ randomly chosen from $\Z_q^k$. We compute $\matr{J}$ as before but now $ck_1,ck_2$ and $ck'$ are computed as follows
\begin{align}
&ck = [\matr{G}_0|\matr{G}_1]_1 =
\begin{cases}
[\vecb{u}_0\matr{A}'_{0}\matr{\Gamma}_1+\vecb{u}_1\matr{A}_0|\vecb{u}_1\matr{A}_1]_1 \text{ with prob. } 1/2\\
[\vecb{u}_0\matr{B}'_{0}\matr{\Gamma}_2+\vecb{u}_1\matr{A}_0|\vecb{u}_1\matr{A}_1]_1 \text{ with prob. } 1/2
\end{cases}
\nonumber \\
&ck_1 = [\matr{H}_0|\matr{H}_1]_2 = [\vecb{u}'_0\matr{A}'_{0}|\vecb{v}'_1\matr{A}'_1]_1 \nonumber \\
&ck_2=  [\matr{G}'_0|\matr{G}'_1]_1 = [\vecb{v}'_0\matr{B}'_0|\vecb{u}'_1\matr{B}'_1]_2 \label{eq:ck-dist}
\end{align}
since $[\vecb{u}]_s\mu$, $\mu\gets \Z_q$, is indistinguishable from a random element in $\GG_s^k$ --- as long as the DDH assumption is hard in $\GG_s$ --- it follows that the new commitment keys are indistinguishable from the original ones.\footnote{Note that now there is no problem with changing the commtiment keys distribution.}

If $[\vecb{e}^\dag]_1 \neq\Com_{ck_1}(\vecb{0})$ we would like to have that $\matr{G}_0 = \vecb{u}_0\matr{A}'_0\matr{\Gamma}_1 + \vecb{u}_1\matr{A}_0$, which happens with probability $1/2$. We have also a similiar situation if $[\vecb{f}^\dag]_1 \neq\Com_{ck_2}(\vecb{0})$ and in both cases we will have a security loss factor of $1/2$. For simplicity we assume that we are in the first case.

We want to prove that $\vecb{c}^\dag\dsum\vecb{e}^\dag\dsum\vecb{f}^\dag\notin\mathrm{Im}(\matr{J})$. Matrix $\matr{J}$ is equal to
$$
\begin{pmatrix}
\vecb{u}_{0}\matr{A}'_0\matr{\Gamma}_1 + \vecb{u}_1\matr{A}_0 & \vecb{u}_{1}\matr{A}_1 & \matr{0} & \matr{0} \\
\vecb{u}'_0\matr{A}'_0\matr{\Gamma}_1  & \matr{0} & \vecb{u}'_1\matr{A}'_1 & \matr{0} \\
\vecb{v}'_0\matr{B}'_0\matr{\Gamma}_2  & \matr{0} & \matr{0} & \vecb{v}'_1\matr{B}'_1
\end{pmatrix}
$$
and $\vecb{d}^\dag\dsum\vecb{e}^\dag\dsum\vecb{f}^\dag$ can be written as
$$
\begin{pmatrix} \vecb{d}^\dag\\ \vecb{e}^\dag \\ \vecb{f}^\dag\end{pmatrix}
=
\begin{pmatrix}
\vecb{u}_{1}\mu_{1} \\
\vecb{u}'_{0}\nu_{0} + \vecb{u}'_1\nu_1\\
\vecb{v}'_0\xi_0 + \vecb{v}_1\xi_1
\end{pmatrix},
\text{ where } \nu_0 \neq 0.
$$
Lets see that $\vecb{d}^\dag\dsum\vecb{e}^\dag\dsum\vecb{f}^\dag$ is not in the immage of $\matr{J}$ by showing that there aren't solutions to $\vecb{d}^\dag\dsum\vecb{e}^\dag\dsum\vecb{f}^\dag=\matr{J}(\vecb{w}_{0}\dsum\vecb{w}_{1}\dsum\vecb{w}_{2}\dsum\vecb{w}_{3})$. Indeed, suppose that
\begin{align}
\begin{pmatrix}
\vecb{u}_{1}\mu_{1} \\
\vecb{u}'_{0}\nu_{0} + \vecb{u}'_1\nu_1\\
\vecb{v}'_0\xi_0 + \vecb{v}_1\xi_1
\end{pmatrix}
=
\begin{pmatrix}
\vecb{u}_0\matr{A}_0\matr{\Gamma}_1\vecb{w}_0 + \vecb{u}_1(\matr{A}_0\vecb{w}_0 + \matr{A}_1\vecb{w}_1)\\
\vecb{u}'_0\matr{A}'_0\matr{\Gamma}_1\vecb{w}_0 + \vecb{u}'_1\matr{A}'_1\vecb{w}_{2}\\
\vecb{v}'_0\matr{B}'_0\matr{\Gamma}_2\vecb{w}_0 + \vecb{v}'_1\matr{B}'_1\vecb{w}_{3}
\end{pmatrix}.
\label{eq:def-li}
\end{align}
Given that $\vecb{u}_{0}$ is linearly independetn from $\vecb{u}_{1}$ and that $\vecb{u}_{0}$ doesn't appear on the left side of the first row of equation (\ref{eq:def-li}), it must hold that $\matr{A}_0\matr{\Gamma}_1\vecb{w}_{0}=0$. Then, the second row is reduced to
$$
\vecb{u}'_{0}\nu_{0} + \vecb{u}'_1\nu_1 = \vecb{u}'_1\matr{A}'_1\vecb{w}_{2}.
$$
Since $\vecb{u}'_0$ is linearly independent from $\vecb{u}'_1$, it must hold that $\nu_0=0$ but this contradicts the fact that $[\vecb{a}']_1\neq\Com_{ck'}(\matr{\Gamma}_1\vecb{x};\vecb{\rho}'_1)$ for all $\vecb{\rho}'_1$. We conclude that $\vecb{d}^\dag\dsum\vecb{e}^\dag\dsum\vecb{f}^\dag$ is not in the immage of $\matr{J}$ and $[\pi-\pi^\dag]_1$ is a proof of a false statement, contradicting the soundness of the QA-NIZK proof system for linear languages
\end{document}
=======
<<<<<<< HEAD
with proof size $2|x|+\Theta(\mathrm{depth}(C))$ elements of a bilinear group. We do so by constructing a QA-NIZK proof system for the language
\[
\mathsf{CircuitSat}_{ck}:=\left\{\begin{array}{l}
([\grkb{\zeta}_1]_1,\ldots,[\grkb{\zeta}_n]_1,C):\exists x_1,\ldots,x_n\in\bits,w_1,\ldots,w_n\in\Z_q \text{ s.t. } \\
C(x)=1 \text{ and } \forall i\in [n]\ [\grkb{\zeta}_i]_1=\GS.\Com_{ck}([x_i]_1;w_i)
\end{array}\right\},
\]
with proof size $\Theta(\mathrm{depth}(C))$.

=======
with proof size $\kappa+\Theta(\mathrm{depth}(C))$ elements of a bilinear group, where $\kappa$ is the size of a proof of knowledge of $\vecb{x}$. In the case of binary circuits, i.e.~$p=2$, we have that $\kappa=2|\vecb{x}|+O(1)$ using the techniques of \cite{AC:GonHevRaf15}. In general, $\kappa$ sould be independent from the circuit.

%We do so by constructing a QA-NIZK proof system for the language
%\[
%\mathsf{CircuitSat}_{ck}:=\left\{\begin{array}{l}
%([\grkb{\zeta}_1]_1,\ldots,[\grkb{\zeta}_n]_1,C):\exists x_1,\ldots,x_n\in\Z_q,\rho_1,\ldots,\rho_n\in\Z_q \text{ s.t. } \\
%C(x)=1 \text{ and } \forall i\in [n]\ [\grkb{\zeta}_i]_1=\GS.\Com_{ck}([x_i]_1;\rho_i)
%\end{array}\right\},
%\]
%with proof size $\Theta(\mathrm{depth}(C))$.

We organize the circuit gates by level, where level $\ell$ is formed by the gates at distance $\ell$ from the output gate. For example, the $d$-th level, where $d:=\mathrm{depth}({C})$, contain the gates whose inputs are only elements from the circuit input $\vecb{x}$ and the $0$-th level contains the unique gate whose output is the output of the circuit.

To each gate we might associate a vector of degree 2 polynomials $\vecb{p}_\ell \in \Z_q^{n_\ell}[W_1,\ldots,W_{m_\ell}]$, where $m_\ell\in\mathbb{N}$ is the number of inputs of level $\ell$ and $n_\ell\in\mathbb{N}$ is the number of outputs (or, equivalently the number of gates) of level $\ell$. Note that it must hold that $\sum_{i<\ell} n_i\geq m_\ell \geq n_{\ell-1}$ ({\color{red} TODO: Check this}). It must hold that for every $\vecb{x}\in\Z_p^m$
$$
C(\vecb{x}) = (\vecb{p}_{d}\circ\vecb{p}_{d-1}\circ\ldots\circ \vecb{p}_0) (\vecb{x}) \text{  \color{red} TODO: I need to add id gates}
$$

We work on asymmetric bilinear groups and our construction is built from the following primitives:
\begin{enumerate}
\item A commitment scheme for vectors in $\Z_q^m$ for wich we can construct a NIZK argument of knowledge of the opening.
\item A commitment scheme for vectors in $\Z_q^m$ with constant-size commitments in $\GG_s^k$, $s=1,2$, for which we can construct a NIZK argument for the following language
$$
\mathcal{L}_{\mathsf{prod},ck_1,ck_2}=\left\{[\vecb{a}]_1,[\vecb{b}]_2,[\vecb{c}]_1:
	\begin{array}{c}
		[\vecb{a}]_1=\mathsf{Com}_{ck_1}(\vecb{x})\text{ and }
		[\vecb{b}]_2=\mathsf{Com}_{ck_2}(\vecb{y})\\
		\Longrightarrow
		[\vecb{c}]_1=\Com_{ck_3}(\vecb{x}\otimes\vecb{y})
	\end{array}\right\},
$$
where $\vecb{x}\in\Z_q^m,\vecb{y}\in\Z_q^n,\vecb{x}\otimes\vecb{y}\in\Z_q^{mn}$, $ck_3=ck_1\otimes ck_2$, and $\otimes$ denote the kroenecker product.
\item A QA-NIZK argument for the language
$$
\mathcal{L} = \left\{[\vecb{c}]_1,[\vecb{c}']_1:
	\begin{array}{l} \text{knowledge of } \vecb{x} \text{ s.t. }
		{[\vecb{c}]_1=\Com_{ck_1\otimes ck_2}(\vecb{x})}
		\Longrightarrow\\
		{[\vecb{c}']_1=\mathsf{Com}_{ck'}(\vecb{x})}
	\end{array}\right\},
$$
\item A QA-NIZK argument for the language
$$
\mathcal{L} = \left\{[\vecb{c}]_1,[\vecb{a}']_1,[\vecb{b}']_2:
	\begin{array}{l}
		\text{knowledge of } \vecb{x} \text{ s.t. }[\vecb{c}]_1=\Com_{ck}(\vecb{x})\\
		\Longrightarrow
		[\vecb{a}']_1=\mathsf{Com}_{ck_1}(\matr{\Gamma}_1\vecb{x})\text{ and }
		[\vecb{b}']_2=\mathsf{Com}_{ck_2}(\matr{\Gamma}_2\vecb{x})
	\end{array}\right\},
$$
\end{enumerate}

\subsubsection{Constant-Size  Multiplicative Homomorphic Commitments.}
Both Groth-Sahai and Pedersen commitments are special cases of the following general commitment scheme
$$
ck:=[\matr{G}]_s=[\matr{G}_0|\matr{G}_1]\in\GG_s^{k\times (n+r)}, \quad \mathsf{Com}_{ck}(\vecb{x};\vecb{\rho})=[\matr{G}_0]_s\vecb{x}+[\matr{G}_1]_s\vecb{\rho}.
$$
Groth-Sahai commitments correspond to the case $k=n+r$, which defines perfectly binding commitments if $\matr{G}$ is invertible, and Pedersen commitments correspond to the case $k=1$, which defines perfectly hiding commitments. We will consider the case $k>1$ which has been called \emph{somewhere statiscally binding} commitments and is a mixture between Groth-Sahai and Pedersen commitments.

With this formulation is easy to derive commitments to $\vecb{x}\otimes\vecb{y}$ from commitments to $\vecb{x}\in\Z_q^m$ and $\vecb{y}\in\Z_q^n$, as follows
$$
\Com_{ck_3}(\vecb{\vecb{x}}\otimes\vecb{y};\vecb{\rho}_3):=\Com_{ck_1}(\vecb{x};\vecb{\rho}_1)\otimes\Com_{ck_2}(\vecb{y};\vecb{\rho}_2),
$$
where $ck_2:=[\matr{H}_0|\matr{H}_2]_1,ck_3=[\matr{G}\otimes\matr{H}]_T$ and
$$\vecb{\rho}_3=\pmatri{\vecb{0}_m\\\vecb{\rho}_1}\otimes\pmatri{\vecb{y}\\\frac{1}{2}\vecb{\rho}_2}+\pmatri{\vecb{x}\\\frac{1}{2}\vecb{\rho}_1}\otimes\pmatri{\vecb{0}_n\\\vecb{\rho}_2}$$ ($\vecb{\rho_3}$ has a different form?).

This approach has the disadvantage that once we compute $[\vecb{c}]_T=\Com_{ck_3}(\vecb{x}\otimes\vecb{y})$ we are stucked in the target group and no more multiplications are possible. But one can still \emph{bootstrap} commitment $[\vecb{c}]_T$ (in some analogy with FHE techniques, when one bootstraps for diminishing the error)  by bringing it to one of the base groups $\GG_s$ and requiring the verifier to check that
$$
e([\vecb{a}]_1,[\vecb{b}]_2)=e([\vecb{c}]_s,[\matr{I}]_{2-s+1}).
$$

Going a step forward, we will have to give two shares of $[\vecb{c}]_s$,  $[\vecb{c}']_1$ and $[\vecb{d}']_2$, such that $\vecb{c}=\vecb{c}'+\vecb{d}'$. We omit the ``primes'' in the shares and now the verifier checks that
$$
e([\vecb{a}]_1,[\vecb{b}]_2)=e([\vecb{c}]_1,[\matr{I}]_{2}) + e([\matr{I}]_{1},[\vecb{d}]_2).
$$

The first share is computed using commitment key $ck_{3,1} := [\matr{G}\otimes\matr{H}-\matr{Z}]_1$ and the second share is computed using commitment key $ck_{3,1} := [\matr{Z}]_2$, for $\matr{Z}\gets\Z_q^{k_1k_2\times mn}$.

\subsubsection{Arguments of Equal Opening.} Given $[\vecb{c}]_1=\Com_{ck}(\vecb{x};\vecb{\rho})$, where $ck = ck_1\otimes ck_2$, we want to show that $[\vecb{c}']_1$ can be also oppened to $\vecb{x}$ but $ck'$ is a random commitment key.

To do so we will give a QA-NIZK argument that $\vecb{c}\dsum\vecb{c}'$ is in the linear span of
$$
\matr{J}:=
\begin{pmatrix}
\matr{G}_0\otimes\matr{H}_0 & \matr{G}_0\otimes\matr{H}_1 & \matr{G}_1\otimes\matr{H}_0& \matr{G}_1\otimes\matr{H}_1 & \matr{0}\\
\matr{G}'_0 & \matr{0} & \matr{0} & \matr{0} & \matr{G}'_0 
\end{pmatrix}
$$
However, the QA-NIZK argument only shows the existence of some $\vecb{w}$ such that $\vecb{c}\dsum\vecb{c}' = \matr{J}\vecb{w}$ but it might be the case that $\vecb{c}'$ still can't be oppened to $\vecb{x}$ --- i.e.~$\vecb{w}$ can't be $\vecb{x}$ appended with some other vector. We will show that this is not the case.

Assume that $[\vecb{c}]_1 = \Com_{ck}(\vecb{x};\vecb{\rho})$ but $[\vecb{c}']_1\neq \Com_{ck'}(\vecb{x};\vecb{\rho}')$ for any $\vecb{\rho}'$, and assume also that the adversary provides a valid proof $[\pi]_1$ for $[\vecb{c}]\dsum[\vecb{c}']_1$. Define $[\vecb{c}^\dag]_1:=\Com_{ck'}(\vecb{x};\vecb{0})$ and note that $\vecb{c}\dsum\vecb{c}^\dag$ is in the immage of $\matr{J}$ and thus we can compute a proof $[\pi^\dag]_1$ for $[\vecb{c}]\dsum[\vecb{c}^\dag]_1$. By the properties of the QA-NIZK arguments for linear spaces, we get that $[\pi-\pi^\dag]_1$ is a proof for $[0]_1\dsum[\vecb{d}]_1$, where $\vecb{d}=\vecb{c}-\vecb{c}^\dag\neq \vecb{0}$. We will show that $\vecb{0}\dsum \vecb{d}$ is not in the immage of $\matr{J}'$, such that $[\matr{J}']_1$ is computationally indistinguishable from $[\matr{J}]_1$.

Let $\vecb{u}_0,\vecb{u}_1,\vecb{v}_0,\vecb{v}_1,\vecb{u}'_0,\vecb{u}'_1$ randomly chosen from $\Z_q^k$. We compute $\matr{J}'$ in the same way that $\matr{J}$ is computed, but now $ck_1,ck_2$ and $ck'$ are computed as follows
\begin{align}
&ck_1 = [\matr{G}_0|\matr{G}_1]_1 = [\vecb{u}_0\matr{A}_{0}|\vecb{u}_1\matr{A}_1]_1 \nonumber \\
&ck_2 = [\matr{H}_0|\matr{H}_1]_2 = [\vecb{v}_0\matr{B}_{0}|\vecb{v}_1\matr{B}_1]_2 \nonumber \\
&ck' =  [\matr{G}'_0|\matr{G}'_1]_1 = [\vecb{u}'_0\matr{A}_0\otimes\matr{B}_0 + \vecb{u}_1\matr{C}_0|\vecb{u}_1\matr{C}_1]_1 \label{eq:ck-dist}
\end{align}
since $[\vecb{u}]_s\mu$, $\mu\gets \Z_q$ is indistinguishable from a random element in $\GG_s^k$ as long as the DDH assumption is hard in $\GG_s$, it follows that the new commitment keys are indistinguishable from the original ones.

There is still a technical problem when using the DDH assumption and computing $[\matr{J}]_1$: when using the DDH assumption in $\GG_2$ to change the distribution of $ck_2$ we can only compute $[\matr{J}]_2$. This problem has already arised and solved in \cite{AC:GonHevRaf15} and we use a similar solution in our final proof system. For the sake of clarity, for this intuitive explanation we just assume that $ck_1,ck_2$ and $ck'$ are sampled from (\ref{eq:ck-dist}) in the real game (although this will make impossible to prove zero-knowledge).

Going back to the problem of whether $\vecb{0}\dsum \vecb{d}$ is in the immage of $\matr{J}$, we get that now this is not the case. Indeed, 
\end{document}

>>>>>>> parent of d790905... asdad
>>>>>>> edd73f541d47428759cdbeee5d7aa21c77966191
The first case is the more general form of a set-membership proof where the set is dynamically chosen. In the second case each instance of the proof system is fixed to a specific set (encoded in the CRS) and is the same notion of the proofs for ``fixed sets'' from Section \ref{sec:bits-applications}. We note that the aggregated set-membership proofs for $S\subset\GG_s$ from Chapter \ref{sec:shuf-rp} are proofs of membership in $\Lang_{ck,S}^n$.

In Section \ref{sec:improved-aZKSMP-intuition}, we start with an intuitive description for the case $S\subset\Z_q$ without aggregation. We note that even in this simpler case, to the best of our knowledge, the shortest non-interactive proof, under falsifiable assumptions and without assuming anything about $S$,\footnote{If $S=[a,b]\subset\Z_q$ and $a<b$ we can use range proofs.} that exists in the literature is the one of Chandran et al.~of size $\Theta(\sqrt{|S|})$.
Our approach is to commit to the binary representation $(b_1,\ldots,b_{\log t})\in\bits^{\log t}$ of the index of the purported $x\in S$, for $S=\{s_1,\ldots,s_t\}$ and where $b_1$ is the least significant bit, to select the the leaves under the paths $(b_{\log t}),(b_{\log t},b_{\log t-1}),\ldots,(b_{\log t},\ldots, b_1)$ in the binary tree whose leaves are (from left to right) $s_1,\ldots,s_t$. In order to keep a logarithmic proof, we commit to the selected leaves using MP commitments from Section \ref{sec:ext-mp} and show, for each $\ell\in[\log t]$, that the leaves under the path $(b_m,\cdots, b_{\ell})$ are equal the leftmost or rightmost, depending of $b_\ell$, leaves under the path $(b_{\log t},\cdots, b_{\ell-1})$. We use these ideas together with a clever usage of QA-NIZK proofs of membership in linear subspaces, Groth-Sahai proofs, and the proof systems from Chapters \ref{sec:agg-asym} and \ref{sec:bits}.
 %In the case of fixed sets $S\subset\Z_q$, Kohlweiss et al.~constructed a non-interactive proof of size $\Theta(1)$, using Boneh-Boyen signatures \cite{PAIRING:RiaKohPre09}.

In Section \ref{sec:log-set-memb-Z} we give a full description of the non-aggregated case and then we show how to extend this result to the case $S\subset\GG_s$. We use the ideas from Section \ref{sec:improved-aZKSMP-intuition} and aggregate many instances using similar techniques to those from Chapter \ref{sec:bits}. We note that, to the best of our knowledge, there is no aggregated proof in the literature (i.e.~all proofs are of size $\Omega(n)$) with the sole exception of our proof from Section \ref{sec:shuf-rp} which is of size $\Theta(|S|)$.  
Our proof bears some similarities with the work of Groth and Kohlweiss \cite{EC:GroKoh15} -- both allow to construct proofs of membership in a set of logarithmic size using the binary encoding of the element index -- but they are in general incomparable. Indeed, Groth and Kohlweiss's construction is on a different setting (interactive, without pairings) and does not support aggregation of many proofs.

There is a straightforward application of the improved aZKSMP. In the proof of a shuffle from Section \ref{sec:shuffle}, the size of the proof that $[\matr{F}]\in\Lang_{ck,S}^n$ can be reduced from $2n+\Theta(1)$ to $\Theta(\log n)$ and thus the total proof size is reduced from $4n+o(n)$ to $2n+o(n)$.
%Furhter, in Section \ref{sec:log-ring-signature} we consider another application of the improved aZKSMP: theoretical $\Theta(\log n)$ ring signatures without random oracles. Although the constants hidden in the asymptotic size of the proof are only polynomially bounded in the security parameter, we interpret this construction as a feasibility result for $\Theta(\log n)$ ring signatures (note that only $\Theta(\sqrt{n})$ ring signatures where known up to this work).


