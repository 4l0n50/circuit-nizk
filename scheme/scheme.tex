Let $\setsize:=|S|$ and \(m:=\log \setsize \). The statement is now \([\grkb{\zeta}_1]=\GS.\Com_{ck_\GS}(x_1;r_1),\ldots,\allowbreak[\grkb{\zeta}_n]_1=\GS.\Com_{ck_\GS}(x_n;r_n)\), for some $n\in\mathbb{N}$, and the prover wants to show that \(x_i=s_{\alpha_i}\), for all \(i\in[n]\) and \(\alpha_i=\sum_{j=1}^m b_{i,j}2^{j-1}+1\), for some $b_{i,1},\ldots,b_{i,m}\in\bits$. We need to reformulate equations (\ref{eq-log-2}) and (\ref{eq-log-3}) to take in count new variables. For \(\ell\in [m],i\in[n]\), define
\begin{align*}
&\vecb{x}_\ell^i:=
\begin{pmatrix}
\vecb{x}^i_{\ell,1}\\
\hline
\vecb{x}^i_{\ell,2}
\end{pmatrix}
:=
\begin{pmatrix}
x^i_{\ell,1}\\\vdots\\x^i_{\ell,2^{\ell-2}}\\
\hline
x^i_{\ell,2^{\ell-2}+1}\\\vdots\\x^i_{\ell,2^{\ell-1}}
\end{pmatrix},
&\vecb{x}^i_{m+1,1} := \begin{pmatrix}
s_1\\\vdots\\s_{t/2}
\end{pmatrix},\text{ and }
&\vecb{x}^i_{m+1,2} := \begin{pmatrix}
s_{t/2+1}\\\vdots\\s_{\setsize }
\end{pmatrix},
\end{align*}
and define new equations for each $\ell\in[m],i\in[n]$
\begin{align}
&\vecb{x}^i_\ell=({1}-{b}_{i,\ell})\vecb{x}^i_{\ell+1,1}+{b}_{i,\ell}\vecb{x}^i_{\ell+1,2},\label{eq-alog-1}\\
&x_i= \vecb{x}^{i}_1\label{eq-alog-2}
\end{align}

Next, we construct an aZKSMP for $S\subset\Z_q$ and in Section~\ref{sec:improved-aZKSMP-group-case} we show how to extend these ideas for the case of fixed \(S\subset\GG_1\).
The construction follows the intuition outlined before but it ``aggregates'' many instances on a single \(\Theta(\log \setsize )\) proof. From a high level this is done as follows.

We will rewrite equation (\ref{eq-alog-1}), which is a system of $m n$ equations, as $m$ equations of the form
\begin{equation}
\vecb{x}\vecb{y}^\top=\pmatri{
    0      & x_1y_2 & \cdots & x_1y_n\\
    x_2y_1 & 0      & \cdots & x_2y_n\\
    \vdots & \vdots & \ddots & \vdots\\
    x_my_1 & x_my_2 & \cdots & 0},
\label{eq-diag}
\end{equation}
where $\vecb{x}\in\Z_q^m,\vecb{y}\in\Z_q^n$ (i.e.~the diagonal of the matrix $\vecb{x}\vecb{y}^\top$ is $\vecb{0}$). We will use similar techniques to those of Chapter \ref{sec:bits} to give a constant size proof for the satisfiability of each of these equations. Therefore, to prove $m$ of these equations we will require $\Theta(m)=\Theta(\log t)$ group elements.

We can compute $\vecb{x}\vecb{y}^\top$ in the ``commitment space'' by means of $[\vecb{c}]_1[\vecb{d}^\top]_2$, where $[\vecb{c}]_1:=\MP.\Com_{ck_1}(\vecb{x};r_1)$ and $[\vecb{d}]_2:=\MP.\Com_{ck_2}(\vecb{y};r_2)$. Indeed, by the definition of MP commitments it holds that
\begin{align*}
[\vecb{c}]_1[\vecb{d}^\top]_2
=&\left(\sum_{i=1}^{m}x_i[\vecb{g}_i]_1+r_1[\vecb{g}_{m+1}]_1\right)\left(\sum_{j=1}^{n}y_j[\vecb{h}_j^\top]_2+r_2[\vecb{h}_{n+1}^\top]_2\right)\\
=&\sum_{i=1}^m\sum_{j=1}^n x_iy_j[\vecb{g}_i\vecb{h}_j^\top]_T+\sum_{i=1}^{m}x_ir_2[\vecb{g}_i\vecb{h}_{n+1}^\top]_T+\sum_{j=1}^{n+1}r_1y_j[\vecb{g}_{m+1}\vecb{h}_j^\top]_T
\end{align*}

Therefore, if the diagonal of $\vecb{x}\vecb{y}^\top$ is $\vecb{0}$, then $[\vecb{c}]_1[\vecb{d}^\top]_2$ is in the space spanned by $\{[\vecb{g}_i\vecb{h}_j^\top]_T:i\neq j \text{ or }i=m+1\text{ or } j=n+1\}$. Similarly as done in Section \ref{sec:bits-intuition}, equation (\ref{eq-diag}) can be proven computing two matrices $[\matr{\Theta}]_1\in\GG_1^{2\times2}$ and $[\matr{\Pi}]_2\in\GG_2^{2\times 2}$ and showing that $[\vecb{c}]_1[\vecb{d}^\top]_2=[\matr{\Theta}]_1[\matr{I}]_2+[\matr{I}]_1[\matr{\Pi}]_2$ and $\matr{\Theta}+\matr{\Pi}\in\Span(\{\vecb{g}_i\vecb{h}_j^\top:i=m+1\text{ or }j=n+1\})$.

For each $\ell\in[m]$, to rewrite the right side of equation (\ref{eq-alog-1}) in the $\vecb{x}\vecb{y}^\top$ form, we observe that
\begin{align*}
\pmatri{\vecb{x}^1_{\ell+1,1}\\\vdots\\\vecb{x}^n_{\ell+1,1}}\left(\pmatri{b_{1,\ell}\\\vdots\\b_{n,\ell}}-\pmatri{1\\\vdots\\1}\right)^\top+
\pmatri{\vecb{x}^1_{\ell+1,1}\\\vdots\\\vecb{x}^n_{\ell+1,2}}\pmatri{b_{1,\ell}\\\vdots\\b_{n,\ell}}^\top
&=\\
\begin{pmatrix}
(1-b_{1,\ell})\vecb{x}^1_{\ell+1,1}+b_{1,\ell}\vecb{x}^1_{\ell+1,2} & \cdots & (1-b_{n,\ell})\vecb{x}^1_{\ell+1,1}+b_{n,\ell}\vecb{x}^1_{\ell+1,2}\\
\vdots & \ddots  & \vdots\\ 
(1-b_{1,\ell})\vecb{x}^n_{\ell+1,1}+b_{1,\ell}\vecb{x}^n_{\ell+1,2} & \cdots & (1-b_{n,\ell})\vecb{x}^n_{\ell+1,1}+b_{n,\ell}\vecb{x}^n_{\ell+1,2}
\end{pmatrix}.&
\end{align*}
If we view the previous matrix as one of size $n\times n$ where each entry is a vector from $\Z_q^{2^{\ell-1}}$, then the diagonal forms the right side of equation (\ref{eq-alog-1}). We rewrite the left side of equation (\ref{eq-alog-1}) as
\begin{align*}
\pmatri{\vecb{x}^1_\ell\\\vdots\\\vecb{x}^n_\ell}\pmatri{1\\\vdots\\1}^\top
=
\begin{pmatrix}
\vecb{x}^1_\ell & \cdots & \vecb{x}^1_\ell\\
\vdots          &        & \vdots         \\
\vecb{x}^n_\ell & \cdots & \vecb{x}^n_\ell
\end{pmatrix}.
\end{align*}
and, again, the diagonal forms the left side of equation (\ref{eq-alog-1}). 

Now we prove that equation (\ref{eq-alog-1}) holds by replacing variables with MP commitments and showing that
\begin{align*}
[\vecb{c}_\ell]_1\left(\sum_{j=1}^n[\vecb{h}_j]_2\right)^\top-[\vecb{c}_{\ell,1}]_1\left([\vecb{d}_\ell]_2-\sum_{j=1}^n[\vecb{h}_j]_2\right)^\top-[\vecb{c}_{\ell,2}]_1[\vecb{d}]_2^\top
=&\\
[\matr{\Theta}]_1[\matr{I}]_2+[\matr{I}]_1[\matr{\Pi}]_2&,
\end{align*}
where
\begin{align*}
&[\vecb{c}_\ell]_1:=\MP.\Com_{ck_\ell}\left(\pmatri{\vecb{x}^1_\ell\\\vdots\\\vecb{x}^n_\ell};r_\ell\right)
&[\vecb{d}_\ell]_1:=\MP.\Com_{ck}\left(\pmatri{b_{1,\ell}\\\vdots\\b_{n,\ell}};t_\ell\right)\\
&[\vecb{c}_{\ell,1}]_1:=\MP.\Com_{ck_\ell}\left(\pmatri{\vecb{x}^1_{\ell+1,1}\\\vdots\\\vecb{x}^n_{\ell+1,1}};r_{\ell,1}\right)
&[\vecb{c}_{\ell,2}]_1:=\MP.\Com_{ck_\ell}\left(\pmatri{\vecb{x}^1_{\ell+1,2}\\\vdots\\\vecb{x}^n_{\ell+1,2}}\right)\\
&ck_\ell:=[\pmatri{\matr{G}_{\ell}^{1}&\cdots&\matr{G}_{\ell}^{n} & \vecb{g}_{\ell,n2^{\ell-1}+1}}]_1
&\matr{G}_{\ell}^{i}:=\pmatri{\vecb{g}_{\ell,(i-1)2^{\ell-1}+1}&\cdots&\vecb{g}_{\ell,i2^{\ell-1}}}\\
&ck:=[\matr{H}]_2 &
\matr{H}:=\pmatri{\vecb{h}_1&\cdots&\vecb{h}_n&\vecb{h}_{n+1}}.
\end{align*}
We need to show that $\matr{\Theta}+\matr{\Pi}$ is in the appropriate space, which is the one without components ``in the diagonal'' or with components in $\vecb{g}_{\ell,n2^{\ell-1}+1}\vecb{h}_j$ or $\vecb{g}_i\vecb{h}_{n+1}$ for any $i\in[n2^{\ell-1}],j\in[n+1]$. However, since we are working with matrices whose entries are vectors in $\Z_q^{2^{\ell-1}}$, we in fact need to show that
$$
\matr{\Theta}+\matr{\Pi}\in\Span(\{\vecb{g}_{\ell,i}\vecb{h}_j^\top:j\in[n+1],i\notin[(j-1)2^{\ell-1}+1,j2^{\ell-1}]\setminus[n2^{\ell-1}+1]\}),
$$
since the indices $i$ and $j$ where $i\in[(j-1)2^{\ell-1}+1,j2^{\ell-1}]$ are those which range over the elements in the diagonal of a matrix whose entries are elements from $\Z_q^{2^{\ell-1}}$.

It is only left to prove the ``aggregated version'' of equation (\ref{eq-log-split}) from the non-aggregated case,  and to prove equation (\ref{eq-alog-2}).
Equation (\ref{eq-log-split}) is proven in the same way as in the non-aggregated case, but enlarging the matrix as consequence of the enlargement of commitment keys. Additionally, we prove equation (\ref{eq-alog-2}) with a proof that
\begin{align*}
[\grkb{\zeta}_1]_1,\ldots,[\grkb{\zeta}_n]_1 \text{ and } [\vecb{c}_1]_1 \text{ open to the same values,}
\end{align*}
using the proof system from Section \ref{sec:aggcommit}.
\iffalse
We prove soundness in the same fashion as the protocols from Chapter \ref{sec:shuf-rp}, that is we guess the index $j^*$ of an element $x_{j^*}\notin S$ in order to choose $\vecb{h}_{j^*}$ linearly independent from the other vectors in $ck$. Similarly as in the non-aggregated case, we will guess the (sub-)path $(b_{j^*,m+1},\ldots, b_{j^*,1})$ (where $b_{j^*,1},\ldots,b_{j^*,m}$ are the uniques openings of, respectively, $[\vecb{d}_1]_2,\ldots,[\vecb{d}]_m$ at position $j^*$) and we will choose, for each $\ell\in[m]$, $\vecb{g}_{\ell,\alpha_{j^*,\ell}}$ linearly independent from the other vectors in $ck_\ell$, where $\alpha_{j^*,\ell}:=(\alpha_{j^*}-1 \mod 2^{\ell-1})+1$. With such choice of the commitment keys, we will be able to prove that $x_{j^*}=s_{\alpha_{j^*}}$ unless we can break some hardness assumption.
Consequently, the total security loss in the reduction will be of a factor of $\frac{2}{nt}$.  
\fi
\subsubsection{The Scheme}

\begin{description}

\item[{\(\algK_1(\gk, ck_\GS)\)}:]
Parse $ck_\GS$ as $[\vecb{u}_1\cat\vecb{u_2}]_1$.
For each \(\ell\in [m]\) let \(\matr{G}_\ell:=\matr{G}_{\ell}^{1}\cat\cdots\cat\matr{G}_{\ell}^{n}\cat\vecb{g}_{\ell,{n2^{\ell-1}+1}}\gets\distlin_1^{n2^{\ell-1}+1,0}\), where
\begin{align*}
&\matr{G}_{\ell}^{i}=
(\matr{G}_{\ell,1}^{i}|\matr{G}_{\ell,2}^{i})
=\\
&(\vecb{g}_{\ell,(i-1)2^{\ell-1}+1}\cdots\vecb{g}_{\ell, (i-1) 2^{\ell-1}+2^{\ell-2}}|\vecb{g}_{\ell,(i-1)2^{\ell-1}+2^{\ell-2}+1}\cdots\vecb{g}_{\ell,i2^{\ell-1}})
\in\Z_q^{2\times 2^{\ell-1}},
\end{align*}
 \(i\in  [n]\), and define \(ck_\ell:=[\matr{G}_\ell]_1\).
Let \(\matr{H}=\begin{pmatrix}\vecb{h}_1&\cdots&\vecb{h}_n&\vecb{h}_{n+1}\end{pmatrix}\gets\distlin_1^{n,0}\) and define \(ck:=[\matr{H}]_2\). 

For each \(\ell\in[m]\), \( i\in [n2^{\ell-1}+1]\), \(j\in  [n+1]\), such that \(i\notin [(j-1)2^{\ell-1}+1,j2^{\ell-1}]\) define matrices
\[\matr{M}^{\ell}_{i,j}:=([\matr{C}_{i,j}^{\ell}]_1,[\matr{D}_{i,j}^{\ell}]_2):=([\vecb{g}_{\ell,i}\vecb{h}_j^{\top}+\matr{T}]_1,[-\matr{T}]_2),\]
For \(\ell\in [m]\), pick \(\matr{T}\gets\Z_q^{2\times 2}\) and let
$$
\mathcal{M}_\ell:=\{\matr{M}^\ell_{i,j}:j\in[n+1],i\notin[(j-1)2^{\ell-1}+1,j2^{\ell-1}]\setminus[n2^{\ell-1}+1]\}$$
and let
$$
\mathcal{C}_\ell:=\{\matr{C}^\ell_{i,j}:j\in[n+1],i\notin[(j-1)2^{\ell-1}+1,j2^{\ell-1}]\setminus[n2^{\ell-1}+1]\}.
$$

Let \(\Pi_\sfsum\) be the proof system for sum in subspace 
(Section~\ref{sec:sum}), \(\Pi_\mathsf{lin}\) the proof system for membership in linear subspaces from Section~\ref{sect:QANIZKlinspace}, \(\Pi_\sfbits\) the proof system for proving that many commitments open to bit-strings from section \ref{sec:matr-bits}, and \(\Pi_\sfcom\)
be an instance of the proof system for equal commitment opening (Section~\ref{sec:aggcommit}).

For each $\ell\in[m]$, let
\(\crs_{\sfsum,\ell} \gets \Pi_\sfsum.\algK_1(\gk, \mathcal{M}_\ell)\).\footnote{We identify
matrices in \(\GG_1^{2 \times 2}\) (respectively in \(\GG_2^{2 \times 2}\)) with vectors in \(\GG_1^{4}\) (resp. in \(\GG_2^{4}\)).}, let \(\crs_{\mathsf{lin},\ell}\gets \Pi_\mathsf{lin}.\algK_1(gk;\allowbreak[\matr{G}_{\ell,\mathsf{split}}]_1,n2^{\ell-1}+3)\), let \(\crs_\sfbits\gets\Pi_\sfbits.\algK_1(gk,[\matr{H}]_2,m)\), and let \(\crs_\sfcom \gets \Pi_\sfcom.\algK_1(\gk, ck_1,CK_\GS,m)\), where
\begin{align*}
&\matr{G}_{\ell,\mathsf{split}}:=\\
&\begin{pmatrix}
\matr{G}^1_{\ell+1,1} & \matr{G}^{1}_{\ell+1,2} & \cdots & \matr{G}^{n}_{\ell+1,1} & \matr{G}^{n}_{\ell+1,2} & \vecb{g}_{\ell+1,n2^\ell+1} & \matr{0}                       & \matr{0}\\
\matr{G}^{1}_{\ell,1}   & \matr{0}              & \cdots & \matr{G}^n_{\ell,n}   & \matr{0}              & \matr{0}                  & \vecb{g}_{\ell,n2^{\ell-1}+1}  & \matr{0}\\
\matr{0}              & \matr{G}^1_{\ell,1}   & \cdots & \matr{0}              & \matr{G}^n_{\ell,n}   & \matr{0}                  & \matr{0}                       & \vecb{g}_{\ell,n2^{\ell-1}+1}
\end{pmatrix},\\
&CK_\GS:=\pmatri{
    [\vecb{u}_1]_1 &        & [\vecb{0}]_1   & [\vecb{u}_2]_1 &        & [\vecb{0}]_1\\
               & \ddots &            &            & \ddots &         \\
    [\vecb{0}]_1   &        & [\vecb{u}_1]_1 & [\vecb{0}]_1   &        & [\vecb{u}_2]_1
}\in\GG_1^{2n\times2n}.
\end{align*}
The common reference string is given by:
\begin{eqnarray*}
\mathsf{crs}&:=&\left( gk, [\matr{G}]_1,
    [\matr{H}]_2, \{\mathcal{M}_\ell,\crs_{\sfsum,\ell},\crs_{\mathsf{lin},\ell}:\ell\in [m]\},\crs_\sfbits,\crs_\sfcom \right).
 \end{eqnarray*}


\item[{\(\algP(\mathsf{crs}, ([\grkb{\zeta}_1]_1, \ldots, [\grkb{\zeta}_n]_1,S), \langle (x_1,\ldots,x_n),(w_1,\ldots,w_n) \rangle)\)}:]
The prover compute commitments
\begin{align*}
&[\vecb{c}_\ell]_1:=\MP.\Com_{ck_\ell}({\vecb{x}_\ell^1}^\top,\ldots,{\vecb{x}_\ell^n}^\top;r_\ell), \text{ for } \ell \in [m],\\
&[\vecb{c}_{\ell,1}]_1:=\MP.\Com_{ck_\ell}({\vecb{x}^1_{\ell+1,1}}^\top,\ldots,{\vecb{x}^n_{\ell+1,1}}^\top;r_{\ell,1}),\\
&[\vecb{c}_{\ell,2}]_1:=\MP.\Com_{ck_\ell}({\vecb{x}^1_{\ell+1,2}}^\top,\ldots,{\vecb{x}^n_{\ell+1,2}}^\top;r_{\ell,2}), \text{ for } \ell\in[m-1]\\
&[\vecb{d}_\ell]_2:=\MP.\Com_{ck}(\vecb{b}_\ell;t_\ell),\text{ for } \ell\in[m]
\end{align*}
 where \(r_\ell,r_{\ell,1},r_{\ell,2},t_j\gets\Z_q\) and the variables \(\vecb{x}^i_\ell,\vecb{x}^i_{\ell,j},\vecb{b}_\ell\) are the ones defined in equation (\ref{eq-alog-1}). The prover computes a proof \(\pi_\sfbits\) that \([\vecb{d}_1]_2,\ldots,[\vecb{d}_m]_2\) open to bit-strings. Then, for \(\ell\in [m]\), the prover pick matrices \(\matr{R}_\ell\gets\Z_q^{2\times 2}\), computes
\begin{align*}
&([\matr{\Theta}_\ell]_1,[\matr{\Pi}_\ell]_2)  := \\
&\quad \sum_{i=1}^n\sum_{j\neq i}\sum_{k=1}^{2^{\ell-1}}(x^i_{\ell,k}-x^i_{\ell+1,k}(1-b_{j,\ell})-x^i_{\ell+1,2^{\ell-1}+k}b_{j,\ell})\matr{M}_{(i-1)2^{\ell-1}+k,j}^{\ell}\\
&\quad+ \sum_{i=1}^n\sum_{k=1}^{2^{\ell-1}}t_\ell(x^i_{\ell+1,k}-x^i_{\ell+1,2^{\ell-1}+k})\matr{M}_{(i-1)2^{\ell-1}+k,n+1}^{\ell} \\
&\quad+ \sum_{j=1}^n (r_\ell-r_{\ell,1}(1-b_{j,\ell})-r_{\ell,2}b_{j,\ell})\matr{M}_{n2^{\ell-1}+1,j}^{\ell} \\
&\quad+(r_{\ell,1}-r_{\ell,2})t_\ell\matr{M}^{\ell}_{n2^{\ell-1}+1,n+1}+([\matr{R}_\ell]_1,[-\matr{R}_\ell]_2),
\end{align*}
where \(r_{1,1}=r_{1,2}=0\), and computes proofs $\pi_{\mathsf{lin},\ell},\pi_{\mathsf{sum},\ell}$ that, respectively,
\begin{align*}
&\begin{pmatrix}
\vecb{c}_{\ell+1}\\\vecb{c}_{\ell,1}\\\vecb{c}_{\ell,2}
\end{pmatrix}\in
\Span(\matr{G}_{\ell,\mathsf{split}})\ (\text{ if }\ell< m), &\matr{\Theta_\ell}+\matr{\Pi_\ell}\in\Span(\mathcal{C}_\ell).
\end{align*}
Finally, it computes a proof \(\pi_\sfcom\) that \(([\grkb{\zeta}_1]_1,\ldots,[\grkb{\zeta}_n]_1)\) and \([\vecb{c}_1]_1\) open to the same value.

The proof is \(\pi:=(\{([\vecb{c}_\ell]_1,[\vecb{c}_{\ell,1}]_1,[\vecb{c}_{\ell,2}]_1,[\vecb{d}_\ell]_2,[\matr{\Theta}_\ell]_1,[\matr{\Pi}_\ell]_2,\pi_{\mathsf{lin},\ell},\pi_{\sfsum,\ell}):\ell\in [m]\},\pi_\sfbits,\pi_\sfcom)\).

\item[{\(\algV(\crs,([\grkb{\zeta}_1]_1, \ldots, [\grkb{\zeta}_n]_1,S),\pi)\)}:]
Let \([\vecb{c}_{m,1}]_1:=\MP.\Com_{ck_m}(s_1,\ldots,s_{\setsize /2};0),[\vecb{c}_{m,2}]:=\MP.\Com_{ck_m}(s_{\setsize /2+1},\ldots,s_{\setsize };0)\). The verifier checks the validity of \(\pi_\sfbits,\pi_\sfcom\) 
and, for each \(\ell\in [m]\), checks the validity of \(\pi_{\mathsf{lin},\ell},\pi_{\sfsum,\ell}\) and of equations
\begin{align}
&[\vecb{c}_\ell]_1\left(\sum_{j=1}^n [\vecb{h}_j]_2\right)^\top-
[\vecb{c}_{\ell,1}]_1\left(\sum_{j=1}^n[\vecb{h}_j]_2-[\vecb{d}_\ell]_2\right)^\top-
[\vecb{c}_{\ell,2}]_1[\vecb{d}_\ell]_2^\top = \nonumber\\
&[\matr{\Theta}_\ell]_1[\matr{I}]_2+[\matr{I}]_1[\matr{\Pi}_\ell]_2. \label{eq-alog-5}
\end{align}
If any of these checks fails, it rejects the proof.

\item[{\(\mathsf{S}_1({gk},ck_\GS)\):}] The simulator receives as input a description of an asymmetric bilinear group \({gk}\) and a GS commitment key $ck_\GS$. It generates and outputs the CRS in the same way as \(\algK_1\), but additionally outputs the simulation trapdoor 
\(\tau:=(\matr{H},\tau_\sfcom,\tau_\sfbits,\{\tau_{\sfsum,\ell},\tau_{\mathsf{lin},\ell}:\ell\in [m]\})\),
where \(\tau_{\sfsum},\tau_{\sfbits},\tau_{\sfsum,\ell},\tau_{\mathsf{lin,\ell}}\) are, respectively, \({\Pi_\sfsum},{\Pi_\sfcom},\Pi_\sfsum,\Pi_\mathsf{lin}\) simulation trapdoors.

\item[{\(\mathsf{S}_2(\crs,([\grkb{\zeta}_1]_1,\ldots,[\grkb{\zeta}_n]_1,S),\tau)\):}] Define \(\vecb{x}_\ell^i:=\vecb{0}\) and \(\vecb{b}_\ell:=\vecb{0}\) for all \(\ell\in [m],i\in[n]\), and computes \([\vecb{c}_\ell]_1, [\vecb{c}_{\ell,1}]_1,[\vecb{c}_{\ell,2}]_1,[\vecb{d}_\ell]_2\) and \([\matr{\Theta}_\ell]_1,[\matr{\Pi}_\ell]_2\), as an honest prover would do (that is, with all variables set to 0).
Finally, simulate proofs \(\pi_\sfcom,\pi_\sfbits,\pi_{\sfsum,\ell},\pi_{\mathsf{lin},\ell}\) using the respective trapdoors.
\end{description}

We prove the following Theorem.

\begin{theorem} \label{theo:bits}
The proof system described above is a QA-NIZK proof system for the language \(\Lang_{ck_\GS,\mathsf{set}}^n\)
 with perfect completeness, computational soundness, and perfect zero-knowledge.
\end{theorem}	

\subsubsection{Completeness}
Completeness follows from completeness of \(\Pi_\sfsum,\Pi_\mathsf{lin},\Pi_\sfbits,\Pi_\sfcom\), and from the fact that equation (\ref{eq-alog-5}) is satisfied for each \(\ell\in [m]\):
\begin{align*}
&\vecb{c}_\ell\left(\sum_{j=1}^n \vecb{h}_j\right)^\top-
\vecb{c}_{\ell,1}\left(\sum_{j=1}^n\vecb{h}_j-\vecb{d}_\ell\right)^\top-
\vecb{c}_{\ell,2}\vecb{d}_\ell^\top &= \\
&\sum_{i=1}^n\sum_{j=1}^n\matr{G}_{\ell}^{i}\vecb{x}^i_{\ell}\vecb{h}_j^\top+\sum_{j=1}^nr_\ell\vecb{g}_{\ell,n2^{\ell-1}+1}\vecb{h}_j^\top
-\sum_{i=1}^n\sum_{j=1}^n\matr{G}_{\ell}^{i}\vecb{x}^i_{\ell+1,1}(1-b_{j,\ell})\vecb{h}_j^\top\\
&+\sum_{i=1}^n\matr{G}_{\ell}^{i}\vecb{x}^i_{\ell+1,1} t_\ell\vecb{h}_{n+1}^\top-\sum_{j=1}^nr_{\ell,1}(1-b_{j,\ell})\vecb{g}_{\ell,n2^{\ell-1}+1}\vecb{h}_j^\top+ r_{\ell,1}t_\ell\vecb{g}_{\ell,n2^{\ell-1}+1}\vecb{h}_{n+1}^\top\\
&-\sum_{i=1}^n\sum_{j=1}^n\matr{G}_{\ell}^{i}\vecb{x}^i_{\ell+1,2}b_{j,\ell}\vecb{h}_j^\top-\sum_{i=1}^n\matr{G}_{\ell}^{i}\vecb{x}^i_{\ell+1,2} t_\ell\vecb{h}_{n+1}^\top-\sum_{j=1}^nr_{\ell,2}b_{j,\ell}\vecb{g}_{\ell,n2^{\ell-1}+1}\vecb{h}_j^\top\\
&- r_{\ell,2}t_\ell\vecb{g}_{\ell,n2^{\ell-1}+1}\vecb{h}_{n+1}^\top &=\\
&\sum_{i=1}^n\sum_{j\neq i}\matr{G}_{\ell}^{i}(\vecb{x}^i_\ell-\vecb{x}^i_{\ell+1,1}(1-b_{j,\ell})-\vecb{x}^i_{\ell+1,2}b_{j,\ell})\vecb{h}_j^\top+\\
&\sum_{i=1}^n\matr{G}_{\ell}^{i}(\vecb{x}^i_{\ell+1,1}-\vecb{x}^i_{\ell+1,2})t_\ell\vecb{h}_{n+1}^\top+\sum_{j=1}^n(r_\ell-r_{\ell,1}(1-b_{j,\ell})-r_{\ell,2}b_{j,\ell})\vecb{g}_{\ell,n2^{\ell-1}+1}\vecb{h}_{j}^\top\\
&+(r_{\ell,1}-r_{\ell,2})t_\ell\vecb{g}_{\ell,n2^{\ell-1}+1}\vecb{h}_{n+1}^\top &=\\
&\sum_{i=1}^n\sum_{j\neq i}\sum_{k=1}^{2^{\ell-1}}(x^i_{\ell,k}-x^i_{\ell+1,k}(1-b_{j,\ell})-x^i_{\ell+1,2^{\ell-1}+k}b_{j,\ell}))\vecb{g}_{\ell,(i-1)2^{\ell-1}+k}\vecb{h}_j^\top\\
&+\sum_{i=1}^n\sum_{k=1}^{2^{\ell-1}}t_\ell(x^i_{\ell+1,k}-x^i_{\ell+1,2^{\ell-1}+k}\vecb{g}_{\ell,(i-1)2^{\ell-1}+k}\vecb{h}_{n+1}^\top\\
&\sum_{j=1}^n(r_\ell-r_{\ell,1}(1-b_{j,\ell})-r_{\ell,2}b_{j,\ell})\vecb{g}_{\ell,n2^{\ell-1}+1}\vecb{h}_{j}^\top+(r_{\ell,1}-r_{\ell,2})t_\ell\vecb{g}_{\ell,n2^{\ell-1}+1}\vecb{h}_{n+1}^\top &=\\
&\matr{\Theta}\matr{I}+\matr{I}\matr{\Pi}.
\end{align*}

\subsubsection{Soundness}

The following theorem guarantees soundness. 
 
\begin{theorem} Let \(\mathsf{Adv}_{{\Pi_\sfset}}(\advA)\) 
be the advantage of an adversary \(\advA\) against the soundness of 
the proof system  described above. There exist PPT adversaries
\(\advD_1,\advD_2,\advB_\sfbits,\advB_\sfcom,\advB_\sfsum,\advB_\mathsf{lin}\) such that 
\begin{align*}
\mathsf{Adv}_{{\Pi_\sfset}}(\advA) \leq 
n \left(\right.
    &\mathsf{Adv}_{\mathcal{L}_1,\Gr}(\advD_1) 
        + \setsize /2\left(4/q
            +  \mathsf{Adv}_{\Pi_\sfbits}(\advB_\sfbits)
            +  \mathsf{Adv}_{\mathcal{L}_1,\Hr}(\advB_2)\right. \\
    &+ \left.\left.\mathsf{Adv}_{{\Pi_\sfcom}}(\advB_\sfcom)
        + m\mathsf{Adv}_{{\Pi_\sfsum}}(\advB_\sfsum)
        + m\mathsf{Adv}_{{\Pi_\mathsf{lin}}}(\advB_\mathsf{lin})\right)\right).
\end{align*}
\label{teo:bitstr-soundness}
\end{theorem}

Recall that, given $b_1,\ldots,b_m\in\bits$, we defined $\alpha:=\sum_{i=1}^mb_i2^{i-1}+1$. Recall also that, given a path $(b_m,\ldots, b_\ell)$ in the binary tree whose leaves are labeled from left to right by $s_1,\ldots,s_t$, we defined $\sfleft:=\sum_{i=\ell}^m b_i2^{i-1}+1$, $\sfright:=\sfleft+2^{\ell-1}-1$, and we defined $\alpha_\ell:=\alpha-\sfleft+1$ the position of $s_\alpha$ relative to the leaves under $s_\sfleft,\ldots,s_\sfright$.

The proof follows from the indistinguishability of the following games:
\begin{itemize}
\item[\(\mathsf{Real}\):] This is the real soundness game. The output is 1 if the adversary submits some \(([\grkb{\zeta}_1]_1,\ldots,[\grkb{\zeta}_n]_1,S)\notin\Lang_{ck_\GS,\mathsf{set}}^n\) and the corresponding proof which is accepted by the verifier.
\item[\(\sfGame_0\):] This identical to \(\mathsf{Real}\), except that \(\algK_1\) does not receive \(ck_\GS\) as a input but
it samples \(ck_\GS\) itself together with its discrete logarithms.
\item[\(\sfGame_1\):] This game is identical to \(\sfGame_0\) except that now it chooses random \(j^*\in[n]\) and it aborts if \(x_{j^*}\notin S\).
\item[\(\sfGame_2\):] This game is identical to \(\sfGame_1\) except that now \(\matr{H}\gets\distlin^{n,j^*}_1\).
\item[\(\sfGame_3\):] This game is identical to \(\sfGame_2\) except that now it defines $b_m:=b_{j^*,m}$ and chooses a random (sub-)path $(b_{m-1},\cdots, b_1)\gets\bits^{m-1}$ (which ignores the first edge) in the tree whose leaves are $s_1,\ldots,s_t$. This game aborts if \((b_{j^*,1},\ldots,b_{j^*,m})\notin\bits^m\) or \((b_1,\ldots, b_{m-1})\neq(b_{j^*,1},\ldots, b_{j^*,m-1})\), where \(b_{j^*,1},\ldots,b_{j^*,m}\) are the openings of \([\vecb{d}_2]_2,\ldots,[\vecb{d}_m]_2\) at coordinate \(j^*\), respectively.
\item[\(\sfGame_4\):] This game is identical to \(\sfGame_3\) except that now \(\matr{G}_\ell\gets\distlin_1^{n2^{\ell-1},\Delta+\alpha_\ell}\), for \(\ell\in [m]\) and $\Delta:=(j^*-1)2^{\ell-1}$.
\end{itemize}

It is obvious that the first two games are indistinguishable. The rest of the argument goes as follows.

\begin{lemma}
\(\Pr\left[ \mathsf{Game}_1(\advA)=1\right]\geq\dfrac{1}{n}\Pr\left[\mathsf{Game}_0(\advA)=1\right].\)
\end{lemma}

\begin{proof}  The probability that
 \(\mathsf{Game}_1(\advA)=1\) is the probability that  a) \(\mathsf{Game}_0(\advA)=1\) and
b)  \(x_{j^*} \notin S\). The view of adversary \(\advA\) is independent of \(j^*\), while, if \(\mathsf{Game_0}(\advA)=1\), then there is at least one index \(j \in [n]\) such that  
such that  \(x_{j} \notin S\). Thus, 
the probability that the event described in b) occurs conditioned on \(\mathsf{Game_0}(\advA)=1\), is greater than or equal to \(1/n\) and the lemma follows.
\end{proof}

\begin{lemma} There exists a\ \(\distlin_1\)-\(\mddh_{\GG_2}\) adversary \(\advD_2\) such that
\(|\Pr\left[\allowbreak\mathsf{Game}_{1}(\advA)\allowbreak=1\right]\linebreak-\Pr\left[\mathsf{Game}_{2}(\advA)=1\right]|\) \(\leq \mathsf{Adv}_{\distlin_1,\ggen_a}(\advD_2).\)
\end{lemma}
\begin{proof}
We construct an adversary \(\advD_2\) that receives 
a challenge \(([\vecb{a}]_2,[\vecb{u}]_2)\) of the 
\(\distlin_1\)-\(\mddh_{\GG_2}\) assumption. From this challenge, \(\advD_2\) just defines the matrix  \([\matr{H}]_2\in\GG_2^{2\times(n+1)}\) as the matrix whose last column is \([\vecb{a}]_2\), the ith column is \([\vecb{u}]_2\), and the rest of the columns are random vectors in the image of \([\vecb{a}]_2\). 
Obviously, when \([\vecb{u}]_2\) is sampled from 
the image of \([\vecb{a}]_2,\) \(\matr{H}\) follows the distribution \(\distlinizeroone\), while if \([\vecb{u}]_2\) is a uniform element of \(\GG^2_2\), \(\matr{H}\) follows the distribution \(\distlin_1^{n,j^*}\). 
 
Adversary \(\advD_2\) samples
\(\matr{G}^{\ell} \gets \distlin_1^{n2^{\ell-1},0}\). Given that \(\advD_2\) does not know the discrete logarithms of \([\matr{H}]_2\), it cannot compute the pairs \((\matr{C}^\ell_{i,j},\matr{D}^\ell_{i,j})\) exactly as in \(\sfGame_0\). Nevertheless, for each \(\ell\in[m],i\in[n2^{\ell-1}+1],j\in[n+1]\) such that $i\notin[(j-1)2^{\ell-1}+1,j2^{\ell-1}]$, it can compute identically distributed pairs by picking \(\matr{T}\gets\Z_q^{2\times 2}\) and defining
\[
([\matr{C}^\ell_{i,j}]_1,[\matr{D}^\ell_{i,j}]_2):=([\matr{T}]_1,\vecb{g}_{\ell,i}[\vecb{h}_j]_2^\top-[\matr{T}]_2).
\]

The rest of the elements of the CRS are honestly computed. When \(\matr{H}\gets\distlin_1^{n,0}\), \(\advD_2\) perfectly simulates \(\sfGame_0\), and when \(\matr{H}\gets\distlin_1^{n,j^*}\), \(\advD_2\) perfectly simulates \(\sfGame_1\), which concludes the proof. 
\end{proof}

\begin{lemma} There exists an adversary \(\advB_\sfbits\) against \(\Pi_\sfbits\) such that
\(\Pr\left[\allowbreak \mathsf{Game}_2(\advA)\allowbreak =1\right]\geq\dfrac{2}{\setsize }(\Pr\left[\mathsf{Game}_3(\advA)=1\right]+\adv_{\Pi_\sfbits}(\advB_\sfbits)).\)
\end{lemma}

\begin{proof}  The probability that
 \(\mathsf{Game}_3(\advA)=1\) is the probability that  a) \(\mathsf{Game}_2(\advA)=1\) and
b) \((b_{j^*,1},\ldots,b_{j^*,m})\notin\bits^m\) or \((b_1,\ldots, b_{m-1}) \neq (b_{j^*,1},\ldots, b_{j^*,m-1})\). If \((b_{j^*,1},\ldots,\allowbreak b_{j^*,m})\notin\bits^m\) we can build an adversary \(\advB_\sfbits\) against \(\Pi_\sfbits\) and thus, the probability that \((b_{j^*,1},\ldots,b_{j^*,m})\in\bits^m\) is less than \(\adv_{\Pi_\sfbits}(\advB_1)\). The view of adversary \(\advA\) is independent of \((b_{1},\ldots, b_{m-1})\), while, if \(\mathsf{Game_2}(\advA)=1\) and \((b_{j^*,1},\ldots,b_{j^*,m})\in\bits^{m}\), then \((b_{j^*,1}\cdots b_{j^*,m-1})\in\bits^{m-1}\). Thus, 
the probability that the event described in b) occurs conditioned on \(\mathsf{Game_2}(\advA)=1\) and \((b_{j^*,1},\ldots,b_{j^*,m})\in\bits^{m}\), is greater than or equal to \(2/\setsize \) and the lemma follows.
\end{proof}

\begin{lemma} There exists a \(\distlin_1\)-\(\mddh_{\GG_1}\) adversary \(\advD_1\) such that
\(|\Pr\left[\mathsf{Game}_{3}(\advA)=1\right]\allowbreak-\Pr\left[\mathsf{Game}_{4}(\advA)=1\right]|\) $\leq
    \mathsf{Adv}_{\distlin_1,\GG_1}(\advD_1).$
\label{lemma:bits2}
\end{lemma}

\begin{proof}
We construct an adversary \(\advD_1\) that receives 
a challenge \(([\vecb{a}]_1,[\vecb{u}]_1)\) of the 
\(\distlin_1\)-\(\mddh_{\GG_1}\) assumption. From this challenge, \(\advD_1\) defines for each \(\ell\in [m]\) the matrix  \([\matr{G}_\ell]_1\) as the matrix whose  \(\Delta+\alpha_\ell\) th column is \([\vecb{u}]_1\), and the rest of the columns are random vectors in the image of \([\vecb{a}]_1\). 
Obviously, when \([\vecb{u}]_1\) is sampled from 
the image of \([\vecb{a}]_1\), \([\matr{G}_\ell]_1\) follows the distribution \(\distlin_1^{n2^{\ell-1},0}\), while if \([\vecb{u}]_1\) is a uniform element of \(\GG^2_1\), \([\matr{G}_\ell]_1\) follows the distribution \(\distlin_1^{n2^{\ell-1},\Delta+\alpha_\ell}\). 
 
The rest of the elements of the CRS are honestly computed. When \([\vecb{u}]_1\) is sampled from the image of \([\matr{a}]_1\), \(\advD_1\) perfectly simulates \(\sfGame_3\), and when \([\vecb{u}]_1\) is uniform, \(\advD_1\) perfectly simulates \(\sfGame_4\), which concludes the proof. 
\end{proof}


\begin{lemma}
There exist adversaries \(\advB_\sfcom\), against the strong soundness of \(\Pi_\sfcom\), \(\advB_\sfsum\), against the soundness of \(\Pi_\sfsum\), and an adversary \(\advB_\mathsf{lin}\) against the soundness of \(\Pi_\mathsf{lin}\), such that \(\Pr[\sfGame_4(\advA)=1]\leq 4/q+ \adv_{\Pi_\sfcom}(\advB_\sfcom)+m\adv_{\Pi_\sfsum}(\advB_\sfsum)+m\adv_{\Pi_\mathsf{lin}}(\advB_\mathsf{lin})\).
\end{lemma}
\begin{proof}
With probability \(1-4/q\), \(\{\vecb{g}_{\ell,\Delta+\alpha_\ell},\vecb{g}_{\ell,n2^{\ell-1}+1}\}\), \(\ell\in [m]\), and \(\{\vecb{h}_{j^*},\allowbreak \vecb{h}_{m+1}\}\) are bases of \(\Z_q^2\),
and, for each \(\ell\in [m],\mu\in\{1,2\}\), we can define \(\tilde{s}_\ell,\tilde{s}_{\ell,\mu},\tilde{r}_\ell,\tilde{r}_{\ell,\mu},b_{j^*,\ell},\tilde{t}_\ell\) as the unique coefficients in \(\Z_q\) such that \(\vecb{c}_\ell=\allowbreak \tilde{s}_\ell\vecb{g}_{\ell,\Delta+\alpha_\ell} + \tilde{r}_\ell \vecb{g}_{\ell,n2^{\ell-1}+1}, \vecb{c}_{\ell,\mu}=\tilde{s}_{\ell,\mu}\vecb{g}_{\ell,\Delta+\alpha_\ell} + \tilde{r}_{\ell,\mu} \vecb{g}_{\ell,n2^{\ell-1}+1},\) and \(\vecb{d}_\ell= b_{j^*,\ell} \vecb{h}_{j^*} + \tilde{t}_\ell \vecb{h}_{n+1}\).

Recall that if \(\sfGame_4(\advA)=1\) then \(x_{j^*}\notin S\). The adversary can win in \(\sfGame_4\) if one of the following events happen:
\begin{description}
\item[\(E_1\):] the adversary breaks soundness of \(\Pi_\sfcom\) and \(x_{j^*}\neq \tilde{s}_1\),
\item[\(E_2\):] the adversary breaks one of the \(m\)  instances of \(\Pi_\sfsum\) and \(\matr{\Theta}_\ell+\matr{\Pi}_\ell\notin\Span(\mathcal{C}_\ell)\),
\item[\(E_3\):] the adversary breaks one of the \(m\) instances of \(\Pi_\sflin\) and \((\vecb{c}_{\ell+1},\vecb{c}_{\ell,1},\vecb{c}_{\ell,2})\notin\Span(\matr{G}_{\ell,\mathsf{split}})\),
\item[\(E_4\):] neither of \(E_1\),\(E_2\), or \(E_3\) happens, but \(x_{j^*}\notin S\) anyway.
\end{description}
By the law of total probabilities, \(\Pr[\sfGame_4(\advA)=1]\leq 4/q+\Pr[E_1]+\Pr[E_2]+\Pr[E_3]+\Pr[E_4]\), and is not hard to see that there exist adversaries \(\advB_\sfcom,\advB_\sfsum,\advB_\mathsf{lin}\) such that \(\Pr[E_1]=\adv_{\Pi_\sfcom}(\advB_\sfcom),\Pr[E_2]=m\adv_{\Pi_\sfsum}(\advB_\sfsum),\) and \(\Pr[E_3]=m\adv_{\Pi_\mathsf{lin}}(\advB_\mathsf{lin})\). Below we will show that \(\Pr[E_4]=0\) (using the same argument used in the non-aggregated case).

We prove by induction on \(\ell\) that \(\tilde{s}_\ell=s_{\alpha}\). If this is the case, the fact that \(\neg E_1\) implies that \(x_{j^*}=\tilde{s}_1=s_{\alpha}\in S\), which finish the proof.

But first note that given a vector \(\vecb{k}\in\Z_q^2\), such that \(\vecb{h}_j^\top\vecb{k}=1\) if \(j=j^*\) and \(0\) if not (which exists since \(\{\vecb{h}_{j^*},\vecb{h}_{n+1}\}\) is a basis of \(\Z_q^2\)), if we multiply equation (\ref{eq-alog-5}) on the right by $\vecb{k}$ we get
$$
[\vecb{c}_\ell]_T-(1-b_{j^*,\ell})[\vecb{c}_{\ell,1}]_T-b_{j^*,\ell}[\vecb{c}_{\ell,2}]_T=[(\matr{\Theta}_\ell+\matr{\Pi}_\ell)\vecb{k}]_T.
$$
The fact that \(\matr{\Theta}_\ell+\matr{\Pi}_\ell\in\Span(\mathcal{C}_\ell)\), \(\vecb{g}_{\ell,i}\in\Span(\vecb{g}_{\ell,n2^{\ell-1}+1})\) if \(i\neq \Delta+\alpha_\ell\), and $\Delta+\alpha_\ell\in[\Delta+1,\Delta+2^{\ell-1}]$, implies that
$$(\matr{\Theta}_\ell+\matr{\Pi}_\ell)\vecb{k} = \sum_{i\in[n2^{\ell-1}+1]\setminus[\Delta+1,\Delta+2^{\ell-1}]}\beta_i\vecb{g}_{\ell,i}=\beta\vecb{g}_{\ell,n2^{\ell-1}+1}$$
for some $\beta_i,{\beta}\in\Z_q$, $i\in[n2^{\ell-1}+1]\setminus[\Delta+1,\Delta+2^{\ell-1}]$.

Therefore, given that we are in the case $b_\ell=b_{j^*,\ell}$, equation (\ref{eq-alog-5}) implies that
$$
[\vecb{c}_\ell]_T=(1-b_{\ell})[\vecb{c}_{\ell,1}]_T+b_{\ell}[\vecb{c}_{\ell,2}]_T+\beta\vecb{g}_{\ell,n2^{\ell-1}+1}.
$$

In the base case ($\ell=m$), the fact that \(\vecb{g}_{m,i}\in\Span(\vecb{g}_{m,n2^{m-1}+1})\), if \(i\neq \Delta+\alpha_m\), implies that 
\begin{align*}
\vecb{c}_m &= (1-b_m)\sum_{i=1}^{2^{m-1}}s_i \vecb{g}_{m,\Delta+i}+b_m\sum_{i=1}^{2^{m-1}}s_{i+2^{m-1}}\vecb{g}_{m,\Delta+i}\\
&= (1-b_m)s_{\alpha_m}\vecb{g}_{m,\Delta+\alpha_m} +b_ms_{\alpha_m+2^{m-1}}\vecb{g}_{m,\Delta+\alpha_m}+\tilde{r}_1\vecb{g}_{m,n2^{m-1}+1}\\
&= (1-b_m)s_{\alpha-\sfleft+1}\vecb{g}_{m,\Delta+\alpha_m} +b_ms_{\alpha-\sfleft+1+2^{m-1}}\vecb{g}_{m,\Delta+\alpha_m}+\tilde{r}_1\vecb{g}_{m,n2^{m-1}+1}\\
&=\begin{cases}
    s_{\alpha-1+1}\vecb{g}_{m,\Delta+\alpha_m}+\tilde{r}_1\vecb{g}_{m,n2^{m-1}+1} & \text{ if } b_m=0 \ (\sfleft=1) \\
    s_{\alpha-(2^{m-1}+1)+1+2^{m-1}}\vecb{g}_{m,\Delta+\alpha_m}+\tilde{r}_1\vecb{g}_{m,n2^{m-1}+1} & \text{ if } b_m=1 \ (\sfleft=2^{m-1}+1) 
\end{cases}
\end{align*}
for some \(\tilde{r}_1\in\Z_q\). In both cases $\vecb{c}_1=s_\alpha\vecb{g}_{m,\Delta+\alpha_m}+\tilde{r}_1\vecb{g}_{m,n2^{m-1}}$.

In the inductive case we assume that \(\vecb{c}_{\ell+1}=s_{\alpha}\vecb{g}_{\ell+1,2\Delta+\alpha_{\ell+1}}+\tilde{r}_{\ell+1}\vecb{g}_{\ell+1,n2^\ell+1}\) and we want to show that $\vecb{c}_\ell = s_\alpha\vecb{g}_{\ell,\Delta+\alpha_\ell}+\tilde{r}_\ell\vecb{g}_{\ell,n2^{\ell-1}+1}$.\footnote{Note that $\matr{G}_{\ell+1}\gets\distlin_1^{n2^\ell, (j^*-1)2^\ell+\alpha_{\ell+1}}$ and thus, the $(j^*-1)2^\ell+\alpha_{\ell+1}=2\Delta+\alpha_{\ell+1}$ th column of $\matr{G}_{\ell+1}$ is l.i.~from the rest.} Since \(\vecb{g}_{\ell+1,\alpha_{\ell+1}}\) is linearly independent from the rest of vectors in \(ck_{\ell+1}\), any solution to 
\begin{equation}
\begin{pmatrix}\vecb{c}_{\ell+1}\\\vecb{c}_{\ell,1}\\\vecb{c}_{\ell,2}\end{pmatrix}=\matr{G}_{\ell,\mathsf{split}}\vecb{w} \label{eq-G-split}
\end{equation}
is equal to \(s_{\alpha}\) at position \(2\Delta+\alpha_{\ell+1}=2\Delta+\alpha_\ell+b_\ell2^{\ell-1}\) as depicted below.
\begin{align*}
\pmatri{\vecb{c}_{\ell+1}\\\vecb{c}_{\ell,1}\\\vecb{c}_{\ell,2}}=
\pmatri{
\cdots & \vecb{g}_{\ell+1,2\Delta+\alpha_\ell} & \cdots  & \vecb{g}_{\ell+1,2\Delta+\alpha_\ell+2^{\ell-1}} & \cdots\\
\cdots & \vecb{g}_{\ell,\Delta+\alpha_\ell}     & \cdots  & \vecb{0}                           & \cdots\\
\cdots & \vecb{0}                        & \cdots  & \vecb{g}_{\ell,\Delta+\alpha_\ell}        & \cdots
}
\pmatri{\vdots\\s_\alpha\\\vdots}
\end{align*}
If $b_{\ell}=0$, by Lemma \ref{lemma:alpha}, $\alpha_{\ell+1}=\alpha_\ell$. Therefore, any solution to equation (\ref{eq-G-split})
 is equal to $s_\alpha$ at position $2\Delta+\alpha_\ell$ and thus $\vecb{c}_{\ell,1} = s_\alpha\vecb{g}_{\ell,\Delta+\alpha_\ell}+\tilde{r}_{\ell,1}\vecb{g}_{\ell,n2^{\ell-1}+1}$.
Equation \ref{eq-alog-5} implies that
\begin{align*}
\vecb{c}_{\ell}=&(1-b_\ell)(s_\alpha\vecb{g}_{\ell,\Delta+\alpha_\ell}+\tilde{r}_{\ell,1}\vecb{g}_{\ell,n2^{\ell-1}+1})+b_\ell\vecb{c}_{\ell,2}+y_\ell\vecb{g}_{\ell,n2^{\ell-1}+1}\\
               =& s_\alpha\vecb{g}_{\ell,\Delta+\alpha_\ell}+(\tilde{r}_{\ell,1}+y_\ell)\vecb{g}_{\ell,n2^{\ell-1}+1}.
\end{align*}
If $b_{\ell}=1$, then $\alpha_{\ell+1}=\alpha_\ell+2^{\ell-1}$ and similarly, $\vecb{c}_{\ell}=s_\alpha\vecb{g}_{\ell,\Delta+\alpha_\ell}+(\tilde{r}_{\ell,2}+y_\ell)\vecb{g}_{\ell,n2^{\ell-1}+1}$.

\end{proof}
\subsubsection{Perfect Zero-Knowledge}
Note that the vectors \([\vecb{c}_\ell],[\vecb{c}_{\ell,1}]_1,[\vecb{c}_{\ell,2}]_1,[\vecb{d}_\ell]_2\) and matrices \([\matr{\Theta}_\ell]_1,[\matr{\Pi}_\ell]_2\), \(1\leq\ell\leq m\), output by the prover and the simulator are, respectively, uniform vectors and uniform matrices conditioned on satisfying equation \ref{eq-alog-5}. This follows from the fact that \(ck,ck_1,\ldots,ck_\ell\) are all perfectly hiding commitment keys and that \([\matr{\Theta}_\ell]_1,[\matr{\Pi}_\ell]_1\) are the unique solutions of equation (\ref{eq-alog-5}) modulo the random choice of \(\matr{R}_\ell\). Finally, the rest of the proof follows from zero-knowledge of \(\Pi_\sfcom,\Pi_\sfbits,\Pi_\sfsum,\) and \(\Pi_\mathsf{lin}\).

\subsection{The case \(S\subset\GG_1\)} \label{sec:improved-aZKSMP-group-case}
We briefly justify that the case \(S\subset\GG_1\) follows directly from the case \(S\subset\Z_q\) when \(S\) is a fixed witness samplable set. That is, there is a fixed set $S$ for each CRS, and there is an efficient algorithm that samples \(s_1,\ldots,s_{\setsize }\in\Z_q\) such that \(S=\{[s_1]_1,\ldots,[s_{\setsize }]_1\}\). %Note that this is the same case of Section~\ref{sec:bits-applications} where the CRS depends on set.

The reason why is not clear how to compute proofs in this setting is that it requires to compute values of the type \([\vecb{s}_i \gamma]_1\), where \([\gamma]_\mu\), \(\mu\in\{1,2\}\), is a group element included in the CRS. The solution is straightforward: use \(s_1,\ldots,s_{\setsize }\) to compute these values and add them to the CRS (with the consequent CRS growth). Therefore, the new CRS contains also, for each $\alpha\in[n],\ell\in[m],i\in[n2^{\ell-1}],j\in[n]$, such that $i\neq(j-1)2^{\ell-1}+\alpha_\ell$:
\begin{align*}
s_\alpha[\vecb{g}_{\ell,(i-1)2^{\ell-1}+\alpha_\ell}]_1 \text{ and }
 s_\alpha([\matr{C}^\ell_{(i-1)2^{\ell-1}+\alpha_\ell,j}]_1,[\matr{D}_{(j-1)2^{\ell-1},j}^\ell]_2).
\end{align*}


