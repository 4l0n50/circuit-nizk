% !TEX root = ../main-circuit-nizk.tex

Let $(\Gen,\Com,\Open)$ an algebraic comitment scheme, $\Pi_\mathsf{PoK}$ a NIZK-AoK proof system for proving knowledge of an opening of a perfectly binding algebriaic commitment, $\Pi_\degtwo$ a QA-NIZK AoK proof systems for proving membership in $\Lang_{\degtwo,ck,ck'}(\vecb{p})$.
\begin{description}
\item[{$\algK_0(gk,k, r)$}:]
Pick $ck_\mathsf{AoK}\gets\Gen(gk,m,m+1,1)$ and pick $ck_\ell\gets\Gen(gk,m_\ell,k,r)$, for $0\leq\ell\leq d$,  from the perfectly hiding distribution. Return
$$
\crs_0 := (gk,ck_\mathsf{AoK},ck_0,\ldots,ck_d).
$$
\item[{$\algK_1(\crs_0, C)$}:] % r and k?
Pick  $\crs_\mathsf{PoK}\gets \Pi_\mathsf{PoK}.\algK_1(gk,ck_\mathsf{AoK})$. Parse $C$ as $\vecb{p}_d\circ\ldots\circ\vecb{p}_1$ such that $\vecb{p}_\ell\in\Z_q^{n_\ell}[X_1,\ldots,X_{m_\ell}]$ and pick $\crs_\ell\gets\Pi_\degtwo.\algK(gk,\vecb{p}_\ell,ck_{\ell-1},ck_\ell)$, for $1\leq\ell\leq d$. Return the refference string defined as $$\crs := (gk,ck_\mathsf{PoK},ck_0,\ldots,ck_{d},\crs_\mathsf{PoK},\crs_1,\ldots,\crs_d)$$

\item[{$\algP(\mathsf{crs}, C, \vecb{x})$}:]
On input $\vecb{x}$ such that $C(\vecb{x})=1$, the prover computes
$$
\vecb{w}_\ell = \vecb{p}_\ell\circ \ldots \vecb{p}_1(\vecb{x}) \text{ for each } 1\leq \ell\leq d
$$
and defines $\vecb{w}_0=\vecb{x}$.
It computes commitments
\begin{align*}
&[\vecb{c}]_1 = \mathsf{KCom}_{ck_\mathsf{AoK}}(\vecb{x};\vecb{\rho}),
&[\vecb{c}_\ell]_1 = \mathsf{Com}_{ck_\ell}(\vecb{w}_\ell;\vecb{\rho}_\ell), \text{ for } 0\leq\ell\leq d,
\end{align*}
where $\vecb{\rho},\vecb{\rho}_0,\ldots,\vecb{\rho}_d\gets\Z_q^r$, and proofs
$$\pi_\ell\gets\Pi_\degtwo.\algP(\crs_\ell,[\vecb{c}_{\ell-1}]_1,[\vecb{c}_\ell]_1,\vecb{w}_{\ell-1},\vecb{\rho}_{\ell-1},\vecb{\rho}_\ell),\text{ for } 1\leq\ell\leq d.$$
Finally, it computes proofs
\begin{align*}
&\pi_{\eq_1} \gets \Pi_\sflin.\algP(\crs_{\sflin_1}, ([\vecb{c}]_1,[\vecb{c}_0]_1),(\vecb{x},\vecb{\rho},\vecb{\rho}_0))\\
&\pi_{\eq_2} \gets \Pi_\sflin.\algP(\crs_{\sflin_2}, [\vecb{c}_d]_1-\Com_{ck_d}(1;0),(0,\vecb{\rho_d})).
\end{align*}
The proof is
$$\pi:=([\vecb{c}]_1,[\vecb{c}_0]_1,\ldots,[\vecb{c}_d]_1,\pi_1,\ldots,\pi_d,\pi_{\eq_1},\pi_{\eq_2}).$$
\item[{\(\algV(\crs,C,\pi)\)}:]
Parse $\pi$ as $([\vecb{c}]_1,[\vecb{c}_0]_1,\ldots,[\vecb{c}_d]_1,\pi_1,\ldots,\pi_d,\pi_{\eq_1},\pi_{\eq_2})$ and check the validity of each of the proofs. Return 0 if any of the ckecks fails, else return 1.

\item[{\(\mathsf{S}_1({gk}\):}] It runs $\algS_1(gk)$ for each $\Pi_\mathsf{AoK},\Pi_\degtwo,\Pi_\sflin$ generating simulated XXX.

\item[{\(\mathsf{S}_2(\crs,([\grkb{\zeta}_1]_1,\ldots,[\grkb{\zeta}_n]_1,S),\tau)\):}] 
\end{description}

We prove the following Theorem.

\begin{theorem} \label{theo:bits}
For any circuit $C$, the proof system described above is a composable NIZK AoK proof system for the language \(\mathsf{CircuitSat}(C)\)
 with perfect completeness, computational soundness, and perfect zero-knowledge.
\end{theorem}	
Perfect completeness follows directly by inspection.
For proving perfect soundness we consider the extractor $\algE$ which on input the corresponding a trapdor $\tau$ together with a proof $\pi=([\vecb{c}]_1,[\vecb{c}_0]_1,\ldots,[\vecb{c}_d]_1,\pi_1,\ldots,\pi_d,\pi_{\eq_1},\pi_{\eq_2})$ for $C$, it outputs $((C,\pi); \vecb{x})$, where $\vecb{x}:=\mathsf{Ext}_{ck_\mathsf{AoK}}(\vecb{c},\tau)$. Computational knowledge soundness follows from the indistinguishibaility of the following games:
\begin{description}
\item[$\sfGame_0$:] This is the real game. The adversary wins if $\advA||\algE$ outputs $((C,\pi),\vecb{x})$ such thaat $C(\vecb{x})=0$ and $\algV(\crs,C,\pi)=1$.
\item[$\sfGame_1$:] This game is exactly as $\sfGame_0$ except that now $ck_0,\ldots,ck_d$ is sampled from the 
\end{description}

\subsubsection{Completeness}
Completeness follows from completeness of \(\Pi_\sfsum,\Pi_\mathsf{lin},\Pi_\sfbits,\Pi_\sfcom\), and from the fact that equation (\ref{eq-alog-5}) is satisfied for each \(\ell\in [m]\):
\begin{align*}
&\vecb{c}_\ell\left(\sum_{j=1}^n \vecb{h}_j\right)^\top-
\vecb{c}_{\ell,1}\left(\sum_{j=1}^n\vecb{h}_j-\vecb{d}_\ell\right)^\top-
\vecb{c}_{\ell,2}\vecb{d}_\ell^\top &= \\
&\sum_{i=1}^n\sum_{j=1}^n\matr{G}_{\ell}^{i}\vecb{x}^i_{\ell}\vecb{h}_j^\top+\sum_{j=1}^nr_\ell\vecb{g}_{\ell,n2^{\ell-1}+1}\vecb{h}_j^\top
-\sum_{i=1}^n\sum_{j=1}^n\matr{G}_{\ell}^{i}\vecb{x}^i_{\ell+1,1}(1-b_{j,\ell})\vecb{h}_j^\top\\
&+\sum_{i=1}^n\matr{G}_{\ell}^{i}\vecb{x}^i_{\ell+1,1} t_\ell\vecb{h}_{n+1}^\top-\sum_{j=1}^nr_{\ell,1}(1-b_{j,\ell})\vecb{g}_{\ell,n2^{\ell-1}+1}\vecb{h}_j^\top+ r_{\ell,1}t_\ell\vecb{g}_{\ell,n2^{\ell-1}+1}\vecb{h}_{n+1}^\top\\
&-\sum_{i=1}^n\sum_{j=1}^n\matr{G}_{\ell}^{i}\vecb{x}^i_{\ell+1,2}b_{j,\ell}\vecb{h}_j^\top-\sum_{i=1}^n\matr{G}_{\ell}^{i}\vecb{x}^i_{\ell+1,2} t_\ell\vecb{h}_{n+1}^\top-\sum_{j=1}^nr_{\ell,2}b_{j,\ell}\vecb{g}_{\ell,n2^{\ell-1}+1}\vecb{h}_j^\top\\
&- r_{\ell,2}t_\ell\vecb{g}_{\ell,n2^{\ell-1}+1}\vecb{h}_{n+1}^\top &=\\
&\sum_{i=1}^n\sum_{j\neq i}\matr{G}_{\ell}^{i}(\vecb{x}^i_\ell-\vecb{x}^i_{\ell+1,1}(1-b_{j,\ell})-\vecb{x}^i_{\ell+1,2}b_{j,\ell})\vecb{h}_j^\top+\\
&\sum_{i=1}^n\matr{G}_{\ell}^{i}(\vecb{x}^i_{\ell+1,1}-\vecb{x}^i_{\ell+1,2})t_\ell\vecb{h}_{n+1}^\top+\sum_{j=1}^n(r_\ell-r_{\ell,1}(1-b_{j,\ell})-r_{\ell,2}b_{j,\ell})\vecb{g}_{\ell,n2^{\ell-1}+1}\vecb{h}_{j}^\top\\
&+(r_{\ell,1}-r_{\ell,2})t_\ell\vecb{g}_{\ell,n2^{\ell-1}+1}\vecb{h}_{n+1}^\top &=\\
&\sum_{i=1}^n\sum_{j\neq i}\sum_{k=1}^{2^{\ell-1}}(x^i_{\ell,k}-x^i_{\ell+1,k}(1-b_{j,\ell})-x^i_{\ell+1,2^{\ell-1}+k}b_{j,\ell}))\vecb{g}_{\ell,(i-1)2^{\ell-1}+k}\vecb{h}_j^\top\\
&+\sum_{i=1}^n\sum_{k=1}^{2^{\ell-1}}t_\ell(x^i_{\ell+1,k}-x^i_{\ell+1,2^{\ell-1}+k}\vecb{g}_{\ell,(i-1)2^{\ell-1}+k}\vecb{h}_{n+1}^\top\\
&\sum_{j=1}^n(r_\ell-r_{\ell,1}(1-b_{j,\ell})-r_{\ell,2}b_{j,\ell})\vecb{g}_{\ell,n2^{\ell-1}+1}\vecb{h}_{j}^\top+(r_{\ell,1}-r_{\ell,2})t_\ell\vecb{g}_{\ell,n2^{\ell-1}+1}\vecb{h}_{n+1}^\top &=\\
&\matr{\Theta}\matr{I}+\matr{I}\matr{\Pi}.
\end{align*}

\subsubsection{Soundness}

The following theorem guarantees soundness. 
 
\begin{theorem} Let \(\mathsf{Adv}_{{\Pi_\sfset}}(\advA)\) 
be the advantage of an adversary \(\advA\) against the soundness of 
the proof system  described above. There exist PPT adversaries
\(\advD_1,\advD_2,\advB_\sfbits,\advB_\sfcom,\advB_\sfsum,\advB_\mathsf{lin}\) such that 
\begin{align*}
\mathsf{Adv}_{{\Pi_\sfset}}(\advA) \leq 
n \left(\right.
    &\mathsf{Adv}_{\mathcal{L}_1,\Gr}(\advD_1) 
        + \setsize /2\left(4/q
            +  \mathsf{Adv}_{\Pi_\sfbits}(\advB_\sfbits)
            +  \mathsf{Adv}_{\mathcal{L}_1,\Hr}(\advB_2)\right. \\
    &+ \left.\left.\mathsf{Adv}_{{\Pi_\sfcom}}(\advB_\sfcom)
        + m\mathsf{Adv}_{{\Pi_\sfsum}}(\advB_\sfsum)
        + m\mathsf{Adv}_{{\Pi_\mathsf{lin}}}(\advB_\mathsf{lin})\right)\right).
\end{align*}
\label{teo:bitstr-soundness}
\end{theorem}

Recall that, given $b_1,\ldots,b_m\in\bits$, we defined $\alpha:=\sum_{i=1}^mb_i2^{i-1}+1$. Recall also that, given a path $(b_m,\ldots, b_\ell)$ in the binary tree whose leaves are labeled from left to right by $s_1,\ldots,s_t$, we defined $\sfleft:=\sum_{i=\ell}^m b_i2^{i-1}+1$, $\sfright:=\sfleft+2^{\ell-1}-1$, and we defined $\alpha_\ell:=\alpha-\sfleft+1$ the position of $s_\alpha$ relative to the leaves under $s_\sfleft,\ldots,s_\sfright$.

The proof follows from the indistinguishability of the following games:
\begin{itemize}
\item[\(\mathsf{Real}\):] This is the real soundness game. The output is 1 if the adversary submits some \(([\grkb{\zeta}_1]_1,\ldots,[\grkb{\zeta}_n]_1,S)\notin\Lang_{ck_\GS,\mathsf{set}}^n\) and the corresponding proof which is accepted by the verifier.
\item[\(\sfGame_0\):] This identical to \(\mathsf{Real}\), except that \(\algK_1\) does not receive \(ck_\GS\) as a input but
it samples \(ck_\GS\) itself together with its discrete logarithms.
\item[\(\sfGame_1\):] This game is identical to \(\sfGame_0\) except that now it chooses random \(j^*\in[n]\) and it aborts if \(x_{j^*}\notin S\).
\item[\(\sfGame_2\):] This game is identical to \(\sfGame_1\) except that now \(\matr{H}\gets\distlin^{n,j^*}_1\).
\item[\(\sfGame_3\):] This game is identical to \(\sfGame_2\) except that now it defines $b_m:=b_{j^*,m}$ and chooses a random (sub-)path $(b_{m-1},\cdots, b_1)\gets\bits^{m-1}$ (which ignores the first edge) in the tree whose leaves are $s_1,\ldots,s_t$. This game aborts if \((b_{j^*,1},\ldots,b_{j^*,m})\notin\bits^m\) or \((b_1,\ldots, b_{m-1})\neq(b_{j^*,1},\ldots, b_{j^*,m-1})\), where \(b_{j^*,1},\ldots,b_{j^*,m}\) are the openings of \([\vecb{d}_2]_2,\ldots,[\vecb{d}_m]_2\) at coordinate \(j^*\), respectively.
\item[\(\sfGame_4\):] This game is identical to \(\sfGame_3\) except that now \(\matr{G}_\ell\gets\distlin_1^{n2^{\ell-1},\Delta+\alpha_\ell}\), for \(\ell\in [m]\) and $\Delta:=(j^*-1)2^{\ell-1}$.
\end{itemize}

It is obvious that the first two games are indistinguishable. The rest of the argument goes as follows.

\begin{lemma}
\(\Pr\left[ \mathsf{Game}_1(\advA)=1\right]\geq\dfrac{1}{n}\Pr\left[\mathsf{Game}_0(\advA)=1\right].\)
\end{lemma}

\begin{proof}  The probability that
 \(\mathsf{Game}_1(\advA)=1\) is the probability that  a) \(\mathsf{Game}_0(\advA)=1\) and
b)  \(x_{j^*} \notin S\). The view of adversary \(\advA\) is independent of \(j^*\), while, if \(\mathsf{Game_0}(\advA)=1\), then there is at least one index \(j \in [n]\) such that  
such that  \(x_{j} \notin S\). Thus, 
the probability that the event described in b) occurs conditioned on \(\mathsf{Game_0}(\advA)=1\), is greater than or equal to \(1/n\) and the lemma follows.
\end{proof}

\begin{lemma} There exists a\ \(\distlin_1\)-\(\mddh_{\GG_2}\) adversary \(\advD_2\) such that
\(|\Pr\left[\allowbreak\mathsf{Game}_{1}(\advA)\allowbreak=1\right]\linebreak-\Pr\left[\mathsf{Game}_{2}(\advA)=1\right]|\) \(\leq \mathsf{Adv}_{\distlin_1,\ggen_a}(\advD_2).\)
\end{lemma}
\begin{proof}
We construct an adversary \(\advD_2\) that receives 
a challenge \(([\vecb{a}]_2,[\vecb{u}]_2)\) of the 
\(\distlin_1\)-\(\mddh_{\GG_2}\) assumption. From this challenge, \(\advD_2\) just defines the matrix  \([\matr{H}]_2\in\GG_2^{2\times(n+1)}\) as the matrix whose last column is \([\vecb{a}]_2\), the ith column is \([\vecb{u}]_2\), and the rest of the columns are random vectors in the image of \([\vecb{a}]_2\). 
Obviously, when \([\vecb{u}]_2\) is sampled from 
the image of \([\vecb{a}]_2,\) \(\matr{H}\) follows the distribution \(\distlinizeroone\), while if \([\vecb{u}]_2\) is a uniform element of \(\GG^2_2\), \(\matr{H}\) follows the distribution \(\distlin_1^{n,j^*}\). 
 
Adversary \(\advD_2\) samples
\(\matr{G}^{\ell} \gets \distlin_1^{n2^{\ell-1},0}\). Given that \(\advD_2\) does not know the discrete logarithms of \([\matr{H}]_2\), it cannot compute the pairs \((\matr{C}^\ell_{i,j},\matr{D}^\ell_{i,j})\) exactly as in \(\sfGame_0\). Nevertheless, for each \(\ell\in[m],i\in[n2^{\ell-1}+1],j\in[n+1]\) such that $i\notin[(j-1)2^{\ell-1}+1,j2^{\ell-1}]$, it can compute identically distributed pairs by picking \(\matr{T}\gets\Z_q^{2\times 2}\) and defining
\[
([\matr{C}^\ell_{i,j}]_1,[\matr{D}^\ell_{i,j}]_2):=([\matr{T}]_1,\vecb{g}_{\ell,i}[\vecb{h}_j]_2^\top-[\matr{T}]_2).
\]

The rest of the elements of the CRS are honestly computed. When \(\matr{H}\gets\distlin_1^{n,0}\), \(\advD_2\) perfectly simulates \(\sfGame_0\), and when \(\matr{H}\gets\distlin_1^{n,j^*}\), \(\advD_2\) perfectly simulates \(\sfGame_1\), which concludes the proof. 
\end{proof}

\begin{lemma} There exists an adversary \(\advB_\sfbits\) against \(\Pi_\sfbits\) such that
\(\Pr\left[\allowbreak \mathsf{Game}_2(\advA)\allowbreak =1\right]\geq\dfrac{2}{\setsize }(\Pr\left[\mathsf{Game}_3(\advA)=1\right]+\adv_{\Pi_\sfbits}(\advB_\sfbits)).\)
\end{lemma}

\begin{proof}  The probability that
 \(\mathsf{Game}_3(\advA)=1\) is the probability that  a) \(\mathsf{Game}_2(\advA)=1\) and
b) \((b_{j^*,1},\ldots,b_{j^*,m})\notin\bits^m\) or \((b_1,\ldots, b_{m-1}) \neq (b_{j^*,1},\ldots, b_{j^*,m-1})\). If \((b_{j^*,1},\ldots,\allowbreak b_{j^*,m})\notin\bits^m\) we can build an adversary \(\advB_\sfbits\) against \(\Pi_\sfbits\) and thus, the probability that \((b_{j^*,1},\ldots,b_{j^*,m})\in\bits^m\) is less than \(\adv_{\Pi_\sfbits}(\advB_1)\). The view of adversary \(\advA\) is independent of \((b_{1},\ldots, b_{m-1})\), while, if \(\mathsf{Game_2}(\advA)=1\) and \((b_{j^*,1},\ldots,b_{j^*,m})\in\bits^{m}\), then \((b_{j^*,1}\cdots b_{j^*,m-1})\in\bits^{m-1}\). Thus, 
the probability that the event described in b) occurs conditioned on \(\mathsf{Game_2}(\advA)=1\) and \((b_{j^*,1},\ldots,b_{j^*,m})\in\bits^{m}\), is greater than or equal to \(2/\setsize \) and the lemma follows.
\end{proof}

\begin{lemma} There exists a \(\distlin_1\)-\(\mddh_{\GG_1}\) adversary \(\advD_1\) such that
\(|\Pr\left[\mathsf{Game}_{3}(\advA)=1\right]\allowbreak-\Pr\left[\mathsf{Game}_{4}(\advA)=1\right]|\) $\leq
    \mathsf{Adv}_{\distlin_1,\GG_1}(\advD_1).$
\label{lemma:bits2}
\end{lemma}

\begin{proof}
We construct an adversary \(\advD_1\) that receives 
a challenge \(([\vecb{a}]_1,[\vecb{u}]_1)\) of the 
\(\distlin_1\)-\(\mddh_{\GG_1}\) assumption. From this challenge, \(\advD_1\) defines for each \(\ell\in [m]\) the matrix  \([\matr{G}_\ell]_1\) as the matrix whose  \(\Delta+\alpha_\ell\) th column is \([\vecb{u}]_1\), and the rest of the columns are random vectors in the image of \([\vecb{a}]_1\). 
Obviously, when \([\vecb{u}]_1\) is sampled from 
the image of \([\vecb{a}]_1\), \([\matr{G}_\ell]_1\) follows the distribution \(\distlin_1^{n2^{\ell-1},0}\), while if \([\vecb{u}]_1\) is a uniform element of \(\GG^2_1\), \([\matr{G}_\ell]_1\) follows the distribution \(\distlin_1^{n2^{\ell-1},\Delta+\alpha_\ell}\). 
 
The rest of the elements of the CRS are honestly computed. When \([\vecb{u}]_1\) is sampled from the image of \([\matr{a}]_1\), \(\advD_1\) perfectly simulates \(\sfGame_3\), and when \([\vecb{u}]_1\) is uniform, \(\advD_1\) perfectly simulates \(\sfGame_4\), which concludes the proof. 
\end{proof}


\begin{lemma}
There exist adversaries \(\advB_\sfcom\), against the strong soundness of \(\Pi_\sfcom\), \(\advB_\sfsum\), against the soundness of \(\Pi_\sfsum\), and an adversary \(\advB_\mathsf{lin}\) against the soundness of \(\Pi_\mathsf{lin}\), such that \(\Pr[\sfGame_4(\advA)=1]\leq 4/q+ \adv_{\Pi_\sfcom}(\advB_\sfcom)+m\adv_{\Pi_\sfsum}(\advB_\sfsum)+m\adv_{\Pi_\mathsf{lin}}(\advB_\mathsf{lin})\).
\end{lemma}
\begin{proof}
With probability \(1-4/q\), \(\{\vecb{g}_{\ell,\Delta+\alpha_\ell},\vecb{g}_{\ell,n2^{\ell-1}+1}\}\), \(\ell\in [m]\), and \(\{\vecb{h}_{j^*},\allowbreak \vecb{h}_{m+1}\}\) are bases of \(\Z_q^2\),
and, for each \(\ell\in [m],\mu\in\{1,2\}\), we can define \(\tilde{s}_\ell,\tilde{s}_{\ell,\mu},\tilde{r}_\ell,\tilde{r}_{\ell,\mu},b_{j^*,\ell},\tilde{t}_\ell\) as the unique coefficients in \(\Z_q\) such that \(\vecb{c}_\ell=\allowbreak \tilde{s}_\ell\vecb{g}_{\ell,\Delta+\alpha_\ell} + \tilde{r}_\ell \vecb{g}_{\ell,n2^{\ell-1}+1}, \vecb{c}_{\ell,\mu}=\tilde{s}_{\ell,\mu}\vecb{g}_{\ell,\Delta+\alpha_\ell} + \tilde{r}_{\ell,\mu} \vecb{g}_{\ell,n2^{\ell-1}+1},\) and \(\vecb{d}_\ell= b_{j^*,\ell} \vecb{h}_{j^*} + \tilde{t}_\ell \vecb{h}_{n+1}\).

Recall that if \(\sfGame_4(\advA)=1\) then \(x_{j^*}\notin S\). The adversary can win in \(\sfGame_4\) if one of the following events happen:
\begin{description}
\item[\(E_1\):] the adversary breaks soundness of \(\Pi_\sfcom\) and \(x_{j^*}\neq \tilde{s}_1\),
\item[\(E_2\):] the adversary breaks one of the \(m\)  instances of \(\Pi_\sfsum\) and \(\matr{\Theta}_\ell+\matr{\Pi}_\ell\notin\Span(\mathcal{C}_\ell)\),
\item[\(E_3\):] the adversary breaks one of the \(m\) instances of \(\Pi_\sflin\) and \((\vecb{c}_{\ell+1},\vecb{c}_{\ell,1},\vecb{c}_{\ell,2})\notin\Span(\matr{G}_{\ell,\mathsf{split}})\),
\item[\(E_4\):] neither of \(E_1\),\(E_2\), or \(E_3\) happens, but \(x_{j^*}\notin S\) anyway.
\end{description}
By the law of total probabilities, \(\Pr[\sfGame_4(\advA)=1]\leq 4/q+\Pr[E_1]+\Pr[E_2]+\Pr[E_3]+\Pr[E_4]\), and is not hard to see that there exist adversaries \(\advB_\sfcom,\advB_\sfsum,\advB_\mathsf{lin}\) such that \(\Pr[E_1]=\adv_{\Pi_\sfcom}(\advB_\sfcom),\Pr[E_2]=m\adv_{\Pi_\sfsum}(\advB_\sfsum),\) and \(\Pr[E_3]=m\adv_{\Pi_\mathsf{lin}}(\advB_\mathsf{lin})\). Below we will show that \(\Pr[E_4]=0\) (using the same argument used in the non-aggregated case).

We prove by induction on \(\ell\) that \(\tilde{s}_\ell=s_{\alpha}\). If this is the case, the fact that \(\neg E_1\) implies that \(x_{j^*}=\tilde{s}_1=s_{\alpha}\in S\), which finish the proof.

But first note that given a vector \(\vecb{k}\in\Z_q^2\), such that \(\vecb{h}_j^\top\vecb{k}=1\) if \(j=j^*\) and \(0\) if not (which exists since \(\{\vecb{h}_{j^*},\vecb{h}_{n+1}\}\) is a basis of \(\Z_q^2\)), if we multiply equation (\ref{eq-alog-5}) on the right by $\vecb{k}$ we get
$$
[\vecb{c}_\ell]_T-(1-b_{j^*,\ell})[\vecb{c}_{\ell,1}]_T-b_{j^*,\ell}[\vecb{c}_{\ell,2}]_T=[(\matr{\Theta}_\ell+\matr{\Pi}_\ell)\vecb{k}]_T.
$$
The fact that \(\matr{\Theta}_\ell+\matr{\Pi}_\ell\in\Span(\mathcal{C}_\ell)\), \(\vecb{g}_{\ell,i}\in\Span(\vecb{g}_{\ell,n2^{\ell-1}+1})\) if \(i\neq \Delta+\alpha_\ell\), and $\Delta+\alpha_\ell\in[\Delta+1,\Delta+2^{\ell-1}]$, implies that
$$(\matr{\Theta}_\ell+\matr{\Pi}_\ell)\vecb{k} = \sum_{i\in[n2^{\ell-1}+1]\setminus[\Delta+1,\Delta+2^{\ell-1}]}\beta_i\vecb{g}_{\ell,i}=\beta\vecb{g}_{\ell,n2^{\ell-1}+1}$$
for some $\beta_i,{\beta}\in\Z_q$, $i\in[n2^{\ell-1}+1]\setminus[\Delta+1,\Delta+2^{\ell-1}]$.

Therefore, given that we are in the case $b_\ell=b_{j^*,\ell}$, equation (\ref{eq-alog-5}) implies that
$$
[\vecb{c}_\ell]_T=(1-b_{\ell})[\vecb{c}_{\ell,1}]_T+b_{\ell}[\vecb{c}_{\ell,2}]_T+\beta\vecb{g}_{\ell,n2^{\ell-1}+1}.
$$

In the base case ($\ell=m$), the fact that \(\vecb{g}_{m,i}\in\Span(\vecb{g}_{m,n2^{m-1}+1})\), if \(i\neq \Delta+\alpha_m\), implies that 
\begin{align*}
\vecb{c}_m &= (1-b_m)\sum_{i=1}^{2^{m-1}}s_i \vecb{g}_{m,\Delta+i}+b_m\sum_{i=1}^{2^{m-1}}s_{i+2^{m-1}}\vecb{g}_{m,\Delta+i}\\
&= (1-b_m)s_{\alpha_m}\vecb{g}_{m,\Delta+\alpha_m} +b_ms_{\alpha_m+2^{m-1}}\vecb{g}_{m,\Delta+\alpha_m}+\tilde{r}_1\vecb{g}_{m,n2^{m-1}+1}\\
&= (1-b_m)s_{\alpha-\sfleft+1}\vecb{g}_{m,\Delta+\alpha_m} +b_ms_{\alpha-\sfleft+1+2^{m-1}}\vecb{g}_{m,\Delta+\alpha_m}+\tilde{r}_1\vecb{g}_{m,n2^{m-1}+1}\\
&=\begin{cases}
    s_{\alpha-1+1}\vecb{g}_{m,\Delta+\alpha_m}+\tilde{r}_1\vecb{g}_{m,n2^{m-1}+1} & \text{ if } b_m=0 \ (\sfleft=1) \\
    s_{\alpha-(2^{m-1}+1)+1+2^{m-1}}\vecb{g}_{m,\Delta+\alpha_m}+\tilde{r}_1\vecb{g}_{m,n2^{m-1}+1} & \text{ if } b_m=1 \ (\sfleft=2^{m-1}+1) 
\end{cases}
\end{align*}
for some \(\tilde{r}_1\in\Z_q\). In both cases $\vecb{c}_1=s_\alpha\vecb{g}_{m,\Delta+\alpha_m}+\tilde{r}_1\vecb{g}_{m,n2^{m-1}}$.

In the inductive case we assume that \(\vecb{c}_{\ell+1}=s_{\alpha}\vecb{g}_{\ell+1,2\Delta+\alpha_{\ell+1}}+\tilde{r}_{\ell+1}\vecb{g}_{\ell+1,n2^\ell+1}\) and we want to show that $\vecb{c}_\ell = s_\alpha\vecb{g}_{\ell,\Delta+\alpha_\ell}+\tilde{r}_\ell\vecb{g}_{\ell,n2^{\ell-1}+1}$.\footnote{Note that $\matr{G}_{\ell+1}\gets\distlin_1^{n2^\ell, (j^*-1)2^\ell+\alpha_{\ell+1}}$ and thus, the $(j^*-1)2^\ell+\alpha_{\ell+1}=2\Delta+\alpha_{\ell+1}$ th column of $\matr{G}_{\ell+1}$ is l.i.~from the rest.} Since \(\vecb{g}_{\ell+1,\alpha_{\ell+1}}\) is linearly independent from the rest of vectors in \(ck_{\ell+1}\), any solution to 
\begin{equation}
\begin{pmatrix}\vecb{c}_{\ell+1}\\\vecb{c}_{\ell,1}\\\vecb{c}_{\ell,2}\end{pmatrix}=\matr{G}_{\ell,\mathsf{split}}\vecb{w} \label{eq-G-split}
\end{equation}
is equal to \(s_{\alpha}\) at position \(2\Delta+\alpha_{\ell+1}=2\Delta+\alpha_\ell+b_\ell2^{\ell-1}\) as depicted below.
\begin{align*}
\pmatri{\vecb{c}_{\ell+1}\\\vecb{c}_{\ell,1}\\\vecb{c}_{\ell,2}}=
\pmatri{
\cdots & \vecb{g}_{\ell+1,2\Delta+\alpha_\ell} & \cdots  & \vecb{g}_{\ell+1,2\Delta+\alpha_\ell+2^{\ell-1}} & \cdots\\
\cdots & \vecb{g}_{\ell,\Delta+\alpha_\ell}     & \cdots  & \vecb{0}                           & \cdots\\
\cdots & \vecb{0}                        & \cdots  & \vecb{g}_{\ell,\Delta+\alpha_\ell}        & \cdots
}
\pmatri{\vdots\\s_\alpha\\\vdots}
\end{align*}
If $b_{\ell}=0$, by Lemma \ref{lemma:alpha}, $\alpha_{\ell+1}=\alpha_\ell$. Therefore, any solution to equation (\ref{eq-G-split})
 is equal to $s_\alpha$ at position $2\Delta+\alpha_\ell$ and thus $\vecb{c}_{\ell,1} = s_\alpha\vecb{g}_{\ell,\Delta+\alpha_\ell}+\tilde{r}_{\ell,1}\vecb{g}_{\ell,n2^{\ell-1}+1}$.
Equation \ref{eq-alog-5} implies that
\begin{align*}
\vecb{c}_{\ell}=&(1-b_\ell)(s_\alpha\vecb{g}_{\ell,\Delta+\alpha_\ell}+\tilde{r}_{\ell,1}\vecb{g}_{\ell,n2^{\ell-1}+1})+b_\ell\vecb{c}_{\ell,2}+y_\ell\vecb{g}_{\ell,n2^{\ell-1}+1}\\
               =& s_\alpha\vecb{g}_{\ell,\Delta+\alpha_\ell}+(\tilde{r}_{\ell,1}+y_\ell)\vecb{g}_{\ell,n2^{\ell-1}+1}.
\end{align*}
If $b_{\ell}=1$, then $\alpha_{\ell+1}=\alpha_\ell+2^{\ell-1}$ and similarly, $\vecb{c}_{\ell}=s_\alpha\vecb{g}_{\ell,\Delta+\alpha_\ell}+(\tilde{r}_{\ell,2}+y_\ell)\vecb{g}_{\ell,n2^{\ell-1}+1}$.

\end{proof}
\subsubsection{Perfect Zero-Knowledge}
Note that the vectors \([\vecb{c}_\ell],[\vecb{c}_{\ell,1}]_1,[\vecb{c}_{\ell,2}]_1,[\vecb{d}_\ell]_2\) and matrices \([\matr{\Theta}_\ell]_1,[\matr{\Pi}_\ell]_2\), \(1\leq\ell\leq m\), output by the prover and the simulator are, respectively, uniform vectors and uniform matrices conditioned on satisfying equation \ref{eq-alog-5}. This follows from the fact that \(ck,ck_1,\ldots,ck_\ell\) are all perfectly hiding commitment keys and that \([\matr{\Theta}_\ell]_1,[\matr{\Pi}_\ell]_1\) are the unique solutions of equation (\ref{eq-alog-5}) modulo the random choice of \(\matr{R}_\ell\). Finally, the rest of the proof follows from zero-knowledge of \(\Pi_\sfcom,\Pi_\sfbits,\Pi_\sfsum,\) and \(\Pi_\mathsf{lin}\).

\subsection{The case \(S\subset\GG_1\)} \label{sec:improved-aZKSMP-group-case}
We briefly justify that the case \(S\subset\GG_1\) follows directly from the case \(S\subset\Z_q\) when \(S\) is a fixed witness samplable set. That is, there is a fixed set $S$ for each CRS, and there is an efficient algorithm that samples \(s_1,\ldots,s_{\setsize }\in\Z_q\) such that \(S=\{[s_1]_1,\ldots,[s_{\setsize }]_1\}\). %Note that this is the same case of Section~\ref{sec:bits-applications} where the CRS depends on set.

The reason why is not clear how to compute proofs in this setting is that it requires to compute values of the type \([\vecb{s}_i \gamma]_1\), where \([\gamma]_\mu\), \(\mu\in\{1,2\}\), is a group element included in the CRS. The solution is straightforward: use \(s_1,\ldots,s_{\setsize }\) to compute these values and add them to the CRS (with the consequent CRS growth). Therefore, the new CRS contains also, for each $\alpha\in[n],\ell\in[m],i\in[n2^{\ell-1}],j\in[n]$, such that $i\neq(j-1)2^{\ell-1}+\alpha_\ell$:
\begin{align*}
s_\alpha[\vecb{g}_{\ell,(i-1)2^{\ell-1}+\alpha_\ell}]_1 \text{ and }
 s_\alpha([\matr{C}^\ell_{(i-1)2^{\ell-1}+\alpha_\ell,j}]_1,[\matr{D}_{(j-1)2^{\ell-1},j}^\ell]_2).
\end{align*}


